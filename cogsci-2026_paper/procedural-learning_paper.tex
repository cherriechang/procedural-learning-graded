% Options for packages loaded elsewhere
% Options for packages loaded elsewhere
\PassOptionsToPackage{unicode}{hyperref}
\PassOptionsToPackage{hyphens}{url}
\PassOptionsToPackage{dvipsnames,svgnames,x11names}{xcolor}
%
\documentclass[
  10pt,
  letterpaper]{article}
\usepackage{xcolor}
\usepackage{amsmath,amssymb}
\setcounter{secnumdepth}{-\maxdimen} % remove section numbering
\usepackage{iftex}
\ifPDFTeX
  \usepackage[T1]{fontenc}
  \usepackage[utf8]{inputenc}
  \usepackage{textcomp} % provide euro and other symbols
\else % if luatex or xetex
  \usepackage{unicode-math} % this also loads fontspec
  \defaultfontfeatures{Scale=MatchLowercase}
  \defaultfontfeatures[\rmfamily]{Ligatures=TeX,Scale=1}
\fi
\usepackage{lmodern}
\ifPDFTeX\else
  % xetex/luatex font selection
\fi
% Use upquote if available, for straight quotes in verbatim environments
\IfFileExists{upquote.sty}{\usepackage{upquote}}{}
\IfFileExists{microtype.sty}{% use microtype if available
  \usepackage[]{microtype}
  \UseMicrotypeSet[protrusion]{basicmath} % disable protrusion for tt fonts
}{}
\makeatletter
\@ifundefined{KOMAClassName}{% if non-KOMA class
  \IfFileExists{parskip.sty}{%
    \usepackage{parskip}
  }{% else
    \setlength{\parindent}{0pt}
    \setlength{\parskip}{6pt plus 2pt minus 1pt}}
}{% if KOMA class
  \KOMAoptions{parskip=half}}
\makeatother
% Make \paragraph and \subparagraph free-standing
\makeatletter
\ifx\paragraph\undefined\else
  \let\oldparagraph\paragraph
  \renewcommand{\paragraph}{
    \@ifstar
      \xxxParagraphStar
      \xxxParagraphNoStar
  }
  \newcommand{\xxxParagraphStar}[1]{\oldparagraph*{#1}\mbox{}}
  \newcommand{\xxxParagraphNoStar}[1]{\oldparagraph{#1}\mbox{}}
\fi
\ifx\subparagraph\undefined\else
  \let\oldsubparagraph\subparagraph
  \renewcommand{\subparagraph}{
    \@ifstar
      \xxxSubParagraphStar
      \xxxSubParagraphNoStar
  }
  \newcommand{\xxxSubParagraphStar}[1]{\oldsubparagraph*{#1}\mbox{}}
  \newcommand{\xxxSubParagraphNoStar}[1]{\oldsubparagraph{#1}\mbox{}}
\fi
\makeatother


\usepackage{longtable,booktabs,array}
\usepackage{calc} % for calculating minipage widths
% Correct order of tables after \paragraph or \subparagraph
\usepackage{etoolbox}
\makeatletter
\patchcmd\longtable{\par}{\if@noskipsec\mbox{}\fi\par}{}{}
\makeatother
% Allow footnotes in longtable head/foot
\IfFileExists{footnotehyper.sty}{\usepackage{footnotehyper}}{\usepackage{footnote}}
\makesavenoteenv{longtable}
\usepackage{graphicx}
\makeatletter
\newsavebox\pandoc@box
\newcommand*\pandocbounded[1]{% scales image to fit in text height/width
  \sbox\pandoc@box{#1}%
  \Gscale@div\@tempa{\textheight}{\dimexpr\ht\pandoc@box+\dp\pandoc@box\relax}%
  \Gscale@div\@tempb{\linewidth}{\wd\pandoc@box}%
  \ifdim\@tempb\p@<\@tempa\p@\let\@tempa\@tempb\fi% select the smaller of both
  \ifdim\@tempa\p@<\p@\scalebox{\@tempa}{\usebox\pandoc@box}%
  \else\usebox{\pandoc@box}%
  \fi%
}
% Set default figure placement to htbp
\def\fps@figure{htbp}
\makeatother





\setlength{\emergencystretch}{3em} % prevent overfull lines

\providecommand{\tightlist}{%
  \setlength{\itemsep}{0pt}\setlength{\parskip}{0pt}}



 
\usepackage[numbers]{natbib}
\bibliographystyle{plainnat}


\usepackage{cogsci}
\usepackage{pslatex}
\hyphenpenalty=10000
\exhyphenpenalty=10000
\makeatletter
\@ifpackageloaded{caption}{}{\usepackage{caption}}
\AtBeginDocument{%
\ifdefined\contentsname
  \renewcommand*\contentsname{Table of contents}
\else
  \newcommand\contentsname{Table of contents}
\fi
\ifdefined\listfigurename
  \renewcommand*\listfigurename{List of Figures}
\else
  \newcommand\listfigurename{List of Figures}
\fi
\ifdefined\listtablename
  \renewcommand*\listtablename{List of Tables}
\else
  \newcommand\listtablename{List of Tables}
\fi
\ifdefined\figurename
  \renewcommand*\figurename{Figure}
\else
  \newcommand\figurename{Figure}
\fi
\ifdefined\tablename
  \renewcommand*\tablename{Table}
\else
  \newcommand\tablename{Table}
\fi
}
\@ifpackageloaded{float}{}{\usepackage{float}}
\floatstyle{ruled}
\@ifundefined{c@chapter}{\newfloat{codelisting}{h}{lop}}{\newfloat{codelisting}{h}{lop}[chapter]}
\floatname{codelisting}{Listing}
\newcommand*\listoflistings{\listof{codelisting}{List of Listings}}
\makeatother
\makeatletter
\makeatother
\makeatletter
\@ifpackageloaded{caption}{}{\usepackage{caption}}
\@ifpackageloaded{subcaption}{}{\usepackage{subcaption}}
\makeatother
\usepackage{bookmark}
\IfFileExists{xurl.sty}{\usepackage{xurl}}{} % add URL line breaks if available
\urlstyle{same}
\hypersetup{
  pdftitle={Procedural Learning with Graded Entropy},
  colorlinks=true,
  linkcolor={blue},
  filecolor={Maroon},
  citecolor={Blue},
  urlcolor={Blue},
  pdfcreator={LaTeX via pandoc}}


\title{Procedural Learning with Graded Entropy}
\author{}
\date{}
\begin{document}
\author{{\bf Cherrie Chang}$^1$ \and {\bf Tianyi Li}$^2$ \and {\bf Joshua Hartshorne}$^1$ \\
  $^1$Department of Communication Sciences and Disorders, Mass General Hospital Institute of Health Professions, Charlestown, MA, USA \\
  $^2$Department of Psychology and Neuroscience, Boston College, Chestnut Hill, MA, USA \\
  cherriechang@gmail.com, lidzr@bc.edu, JKHartshorne@gmail.com}
\maketitle


\section{Abstract}\label{abstract}

Your abstract text here. The abstract should be one paragraph. Following
the abstract should be keywords.

\textbf{Keywords:} procedural learning, implicit learning, statistical
learning

\section{Introduction}\label{introduction}

Your introduction text here. You can cite references using
\texttt{@NewellSimon1972a} or \texttt{{[}@ChalnickBillman1988a{]}}.

\section{Method}\label{method}

\subsection{Participants}\label{participants}

We recruited 251 participants (133 female, 118 male) to complete a
web-based experiment using the online crowd-sourcing platform Prolific,
screening for participants with an approval rating above 95\% from
previous studies and limiting participation to be on desktop or laptop
computers. Participants ranged in age from 18 to 75 years (\(\mu\)=35.6,
\(\sigma\)=11.9), representing 32 nationalities and 20 primary
languages. Participants were compensated at \$12.00 per hour, with
median completion time varying by condition (\textasciitilde20-40
minutes). All participants provided informed consent prior to the
experiment. The study was approved by the Institutional Review Board at
the MGH Institute of Health Professions.

\subsection{Probabilistic Serial Reaction Time
Task}\label{probabilistic-serial-reaction-time-task}

Like other serial reaction time task designs, our experiment tasks
participants to respond to a visual stimulus that appears in one of
several evenly-spaced positions on screen as quickly and accurately as
possible by pressing a corresponding key on the keyboard. In our
implementation, this visual stimulus is a mole garbed in a bright red
bib and matching sunglasses (Figure~\ref{fig-example-trial-screen}); and
a schematic of each position-to-keyboard mapping for each condition is
shown in (Table). Participants were evenly divided into 5 groups of 50
per condition, with the conditions differing in the number of positions
(4, 5, 6, 7, 8) the mole can appear in.

\begin{figure}

\centering{

\includegraphics[width=1\linewidth,height=\textheight,keepaspectratio]{"../assets/example-trial-screen.png"}

}

\caption{\label{fig-example-trial-screen}An example trial screen in the
8-position condition of the experiment}

\end{figure}%

A participant is first given a text-based tutorial accompanied by an
animated demonstration on which key to press in response to each
position, then instructed to work through a set of practice trials where
each position is visited twice in random order. After the practice
section, the participant moves on to complete 20 blocks of trials, each
separated with a self-paced break. Each block consists of trials 10x the
number of positions in the participant's assigned condition. This design
choice results in longer blocks for conditions with more positions, but
ensures each position is visited an equal number of times on average
across conditions. In both practice and main trials, participants are
given feedback on correctness via a pop-up short message (a checkmark
vs.~``Try again!'' + an error tone), and allowed to retry each failed
trial until they respond correctly. Following standard SRTT protocol,
there is also a 120ms response-stimulus interval (RSI) between trials
{[}CITE{]}. We find during piloting that allowing retries, combined with
the 120ms RSI, prevents participants from making compensatory errors,
where they unintentionally press the wrong key in the next trial in an
attempt to correct their current trial. We record response time,
keyboard response and correctness for each trial. After each block,
participants are given adaptive feedback based on their accuracy and
speed. At the end of the experiment, participants complete a brief
questionnaire probing their explicit awareness of any patterns in the
mole's appearances (Table).

In addition to our dapper mole, a number of design choices were made to
facilitate participant understanding and engagement. During the practice
phase, each position on screen is labelled with its corresponding key
character (e.g.~``A'', ``S'', ``D'') to help familiarize participants
with the keyboard mappings. These key labels disappear in the main
trials to prevent explicitly encoding of the sequence of positions via
their character labels, but reappear during retry of failed trials. To
make the correspondence between the positions on screen and keyboard
keys as automatic and intuitive as possible, we chose to use
conventional QWERTY-based finger layouts for the key mappings across
conditions (Table). We found during piloting that while this decision
helped reduce cognitive load from learning novel finger placements, the
larger gap between keys ``F'' and ``J'' on the keyboard introduced a
discrepancy between the evenly-spaced on-screen positions and the
unevenly spaced keyboard keys. This made participants more prone to
making errors in the positions towards the middle of the screen compared
to those on the outer edges, especially when number of positions
increases {[}TODO: maybe run stats on pilot to find this effect{]}; and
was not mitigated by shifting either hand's placement to close the extra
gap, because the positions were still corresponding half to one hand and
half to the other. After further piloting testing different visual aids,
we settled on adding two hands visually to the bottom of the screen,
each finger aligned with its target position (Figure). We received
verbal feedback that this visual aid helped participants better map the
spatial layout of the on-screen positions to their corresponding keys.
Lastly, the visual design of the ``positions'' on screen mimicked blank
3D keyboard keys that light up in pink when the mole appears on top, and
look visibly ``pressed down'' when the participant presses the
corresponding keyboard key. This design choice links the on-screen
positions diegetically to the physical keyboard keys to maximally reduce
mental distance between the two. This is important to enforce as much
learning via implicit motor memory as possible, rather than via
visual-spatial memory of the on-screen positions.

\subsection{Transition Matrices}\label{transition-matrices}

Our experiment deviates from other serial reaction time tasks in how the
sequence of positions and its probabilistic structure is constructed. In
each run of the experiment, we use a first-order Markov process to
generate the sequence of positions the mole appears in. This process is
defined by a transition matrix, where each entry in the matrix defines
the probability of transitioning from one position to another. We
designed five different transition matrices of sizes 4x4, 5x5, 6x6, 7x7,
and 8x8 respectively corresponding to the five conditions in our
experiment. These ``original'' matrices were constructed to have graded
and increasing levels of entropy across rows, such that position 1 has
lowest entropy-having the most predictable next-position-transitions,
followed by the second, with the last position, position n, having
highest entropy. For example, in the 4x4 matrix, position 1 has a high
probability (\textasciitilde0.97) of transitioning to position 2, a low
probability (\textasciitilde0.03) of transitioning to position 1, and
never transitions to position 3 or 4; while position 4 has a more
uniform distribution of transition probabilities to all other positions
(to 1: \textasciitilde(0.39), to 2: \textasciitilde(0.03), to 3:
\textasciitilde(0.24), to 4: \textasciitilde(0.35)). This gives rise to
a gradient of entropy values across the positions, which results in a
richer, more varied distributed statistical structure underlying the
sequence of positions the mole appears in, allowing us to investigate
how the procedural learning system differentially picks up varying
levels of predictability in a given statistical structure.

Several design criteria/constraints were imposed on the construction of
these original transition matrices. First, these matrices had to be
doubly stochastic to ensure uniform visitation to each position over
time, preventing confounding effects from position frequency on
learning. Second, self-transitions (e.g.~position 1 to position 1)
should have near-zero probabilities to reduce data loss, as data from
repeated trials have been historically thrown out from analysis in
serial reaction time tasks to prevent confounding motor effects from
repeating the same keypress {[}CITE{]}. Third, the graded entropy levels
across the positions should translate to graded probabilities across
bigrams as well, as having more fine-grained, distributed levels of
transition probabilities is our experiment's key distinction from other
SRTT tasks, such as the ASRT paradigm which only has 2 levels of trigram
probability. Fourth, the increase in entropy across positions
(i.e.~rows) should be significant between one another and as linear as
possible to ensure smooth a smooth gradient with meaningfully distinct
levels in predictability across positions.

Strictly satisfying all these constraints at once turned out to be not
only non-trivial, but mathematically impossible for our range of
n\_positions (i.e.~matrix sizes), so we wrote an optimization algorithm
that explores the candidate space for each matrix size, treating these
constraints as soft penalties to find a global minimum of the total
constraint violation, balancing trade-offs among the competing
constraints.

The five original matrices we constructed using this algorithm had these
properties: 1) near fully doubly stochastic---all row and column sums
sum to 1 (±0.0001) {[}TODO: consider not hardcoding{]}; 2) approach a
uniform stationary distribution after 7 timesteps {[}TODO: consider not
hardcoding{]} where the stationary visitation for each state (position)
are within 0.001 of each other; 3) had a graded entropy structure across
states---each matrix had a positional entropy range of 0.061-0.181 bits
for position 1 (\(\mu\)=0.107, \(\sigma\)=0.048), and 1.695-2.835 bits
for position n (\(\mu\)=2.319, \(\sigma\)=0.451). Within a matrix each
position increases in entropy from the previous by 0.295-0.699 bits
(\(\mu\)=0.442, \(\sigma\)=0.085); and 4) avoided self loops as much as
possible---all of the matrices had \textless0.08 {[}TODO: consider not
hardcoding{]} probabilities in the left diagonal up until the last two
positions. Within each matrix size condition, all participants
experience the same underlying entropy gradient across positions, but
the mapping between screen positions and entropy levels is randomized
through a double permutation procedure that shuffles the original matrix
(of that size) using the same random permutation. This preserves the
statistical properties of the original matrix---transition probabilities
and entropy values---but decouples entropy levels from specific screen
positions. The position sequence for each participant is subsequently
generated using their individual shuffled transition matrix, so the
sequence each participant experiences is different both within and
between matrix size conditions.

\subsection{Hypotheses}\label{hypotheses}

\begin{quote}
\textbf{H0. Procedural learning effect.} We predict participants to show
implicit learning of the statistical structure across blocks, evidenced
by a) decrease in response time across blocks and b) increase in
accuracy across blocks.
\end{quote}

\begin{quote}
\textbf{H1a. Surprisal effect.} In each experiment run, positions differ
in how many next-positions are possible and how evenly distributed the
transition probabilities are among them, which also means different
next-position/bigram combinations for a given first position yield
different levels of predictability and thus surprisal to the
participant. We predict that participants will respond faster and more
accurately on bigram transitions that are higher probability.
\end{quote}

\begin{quote}
\textbf{H1b. Previous positional entropy effect.} In each experiment
run, positions differ in their transition entropy---how many
next-positions are possible and how evenly distributed the transition
probabilities are among them. We predict that this influences how easily
participants can predict the next position based on a given position, so
participants will respond a) faster and b) more accurately on trials
where the previous trial has lower entropy.
\end{quote}

\begin{quote}
\textbf{H1c. Interaction effect between surprisal and previous
positional entropy.} The effect of surprisal on rt will be modulated by
entropy, such that surprisal effects are stronger at low-entropy
(predictable) positions and weaker at high-entropy (uncertain)
positions. This is because entropy determines prediction confidence. In
low-entropy contexts, participants form strong, specific expectations,
making prediction errors (high surprisal) more costly. In high-entropy
contexts, predictions are weaker and more distributed, attenuating
surprisal effects.
\end{quote}

\begin{quote}
\textbf{H2a. Prediction error reflecting surprisal effect.} We predict
that as participants implicitly learn the underlying statistical
distribution, the mistakes they make reflect this learning, i.e.~across
blocks, the mistakes they make will be a) increasingly on low
probability bigram transitions and b) the false transitions they make
will increasingly be higher probability bigram transitions\ldots{}
\end{quote}

\begin{quote}
\textbf{H2b. Prediction error reflecting entropy effect.} \ldots and
they will also be increasingly limited to higher previous positional
entropy trials.
\end{quote}

\begin{quote}
\textbf{H3. Moderation from number of positions.} We predict that the
more positions there are to handle, the more difficult it is to learn
the underlying statistical distribution, so participants with a larger
number of positions will have a gentler learning slope, higher
mean/median RT and also show worse performances/take more blocks to
reach the same level of performance (in terms of both RT and accuracy)
on the effects predicted in H1 (and H2).
\end{quote}

\emph{Note: Matrix size is confounded with entropy range, so effects may
reflect entropy distribution rather than number of positions per se.}

\section{Statistical Analysis}\label{statistical-analysis}

\subsection{Variables and Exclusion
Criteria}\label{variables-and-exclusion-criteria}

The main independent variables are surprisal and previous positional
entropy. Surprisal is defined as the negative log probability of the
actual next-position given the current position {[}TODO: math{]},
derived from the transition matrix used to generate the sequence for
that participant. Previous positional entropy is defined as the Shannon
entropy of the transition probabilities from the previous position, also
derived from the transition matrix {[}TODO: math{]}. Both surprisal and
previous positional entropy are mean-centered within each matrix size
condition to facilitate interpretation of main effects and interactions
in subsequent models. We also create a binary variable indicating
whether the current position is a repetition of the previous position
(i.e.~self-transition). This variable is included as a covariate in
subsequent models to control for potential motor effects from repeating
the same keypress.

The main dependent variables are response time (RT) and accuracy
(proportion of correct responses) on non-practice/main trials. Incorrect
trials, trials with RT \textgreater1000ms RT, and trials with RTs
greater than 3 median absolute deviations above the participant's median
RT (outliers) are excluded. On the participant level, participants with
overall accuracy below 85\%, median RT \textgreater1000ms, or with
outliers making up more than 30\% of total trials are excluded from
analysis. We also discard data from the first trial of each block to
account for blocks being implicitly treated as ``restarts'' and not
following an existing probability distribution by participants. RTs are
log-transformed to reduce skewness.

\subsection{Modeling Approach}\label{modeling-approach}

We use linear mixed effects regression (LMER) to model log-transformed
RTs and generalized linear mixed effects regression (GLMER) with a
binomial link function to model accuracy. Models are fit using the
\texttt{lmerTest} package in R. We first test for the learning effect
across blocks (H0), then examine the effects of surprisal and previous
positional entropy (H1 \& H2) in the final block, pruning effects where
applicable to narrow in on the best-fitting effects structure before
analyzing learning trajectories across blocks, and finally test for
moderation by number of positions/matrix size(H3).

Models are initially fit with the maximal random effects structure
supported by the design: random intercepts for participant and position,
and random slopes for surprisal, previous positional entropy, their
interaction, and position repetition by participant. Where maximal
models fail to converge, we use bounded linear mixed models
(\texttt{blmer}/\texttt{bglmer}) as a first remedy. If singularity
persists, we iteratively simplify the random effects structure, removing
random slopes one at a time and retaining the most complex structure
that converges without singularity. Among converging models, we select
the one with the lowest AIC.

For H1 and H2, we first establish the fixed effects structure using only
the final block of trials, then extend to learning trajectories across
blocks. Starting from a base model including surprisal, previous
positional entropy, their interaction, and repetition as fixed effects
(with the selected random effects structure), we use AIC-based model
comparison to prune fixed effects. We first test whether repetition is
significant by comparing the base model against one in which it is
removed; if removing it does not increase AIC, we drop it from the base
model. We then test whether the interaction term between surprisal and
previous positional entropy is significant by comparing the current base
model against one in which the interaction term is removed. If the
interaction is not significant, we test each main effect individually by
comparing models that omit surprisal or previous positional entropy as
main effects. An effect is retained if removing it increases AIC.

\subsection{H0. Learning Effect}\label{h0.-learning-effect}

\section{Results}\label{results}

\subsection{H0}\label{h0}

\subsection{H1}\label{h1}

\subsection{H2}\label{h2}

\subsection{H3}\label{h3}

\section{Discussion}\label{discussion}

\section{References}\label{references}

\renewcommand{\bibsection}{}
\bibliography{CogSci_Template.bib}





\end{document}
