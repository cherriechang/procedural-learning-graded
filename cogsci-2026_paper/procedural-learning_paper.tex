% Options for packages loaded elsewhere
% Options for packages loaded elsewhere
\PassOptionsToPackage{unicode}{hyperref}
\PassOptionsToPackage{hyphens}{url}
\PassOptionsToPackage{dvipsnames,svgnames,x11names}{xcolor}
%
\documentclass[
  10pt,
  letterpaper]{article}
\usepackage{xcolor}
\usepackage{amsmath,amssymb}
\setcounter{secnumdepth}{-\maxdimen} % remove section numbering
\usepackage{iftex}
\ifPDFTeX
  \usepackage[T1]{fontenc}
  \usepackage[utf8]{inputenc}
  \usepackage{textcomp} % provide euro and other symbols
\else % if luatex or xetex
  \usepackage{unicode-math} % this also loads fontspec
  \defaultfontfeatures{Scale=MatchLowercase}
  \defaultfontfeatures[\rmfamily]{Ligatures=TeX,Scale=1}
\fi
\usepackage{lmodern}
\ifPDFTeX\else
  % xetex/luatex font selection
\fi
% Use upquote if available, for straight quotes in verbatim environments
\IfFileExists{upquote.sty}{\usepackage{upquote}}{}
\IfFileExists{microtype.sty}{% use microtype if available
  \usepackage[]{microtype}
  \UseMicrotypeSet[protrusion]{basicmath} % disable protrusion for tt fonts
}{}
\makeatletter
\@ifundefined{KOMAClassName}{% if non-KOMA class
  \IfFileExists{parskip.sty}{%
    \usepackage{parskip}
  }{% else
    \setlength{\parindent}{0pt}
    \setlength{\parskip}{6pt plus 2pt minus 1pt}}
}{% if KOMA class
  \KOMAoptions{parskip=half}}
\makeatother
% Make \paragraph and \subparagraph free-standing
\makeatletter
\ifx\paragraph\undefined\else
  \let\oldparagraph\paragraph
  \renewcommand{\paragraph}{
    \@ifstar
      \xxxParagraphStar
      \xxxParagraphNoStar
  }
  \newcommand{\xxxParagraphStar}[1]{\oldparagraph*{#1}\mbox{}}
  \newcommand{\xxxParagraphNoStar}[1]{\oldparagraph{#1}\mbox{}}
\fi
\ifx\subparagraph\undefined\else
  \let\oldsubparagraph\subparagraph
  \renewcommand{\subparagraph}{
    \@ifstar
      \xxxSubParagraphStar
      \xxxSubParagraphNoStar
  }
  \newcommand{\xxxSubParagraphStar}[1]{\oldsubparagraph*{#1}\mbox{}}
  \newcommand{\xxxSubParagraphNoStar}[1]{\oldsubparagraph{#1}\mbox{}}
\fi
\makeatother

\usepackage{color}
\usepackage{fancyvrb}
\newcommand{\VerbBar}{|}
\newcommand{\VERB}{\Verb[commandchars=\\\{\}]}
\DefineVerbatimEnvironment{Highlighting}{Verbatim}{commandchars=\\\{\}}
% Add ',fontsize=\small' for more characters per line
\usepackage{framed}
\definecolor{shadecolor}{RGB}{241,243,245}
\newenvironment{Shaded}{\begin{snugshade}}{\end{snugshade}}
\newcommand{\AlertTok}[1]{\textcolor[rgb]{0.68,0.00,0.00}{#1}}
\newcommand{\AnnotationTok}[1]{\textcolor[rgb]{0.37,0.37,0.37}{#1}}
\newcommand{\AttributeTok}[1]{\textcolor[rgb]{0.40,0.45,0.13}{#1}}
\newcommand{\BaseNTok}[1]{\textcolor[rgb]{0.68,0.00,0.00}{#1}}
\newcommand{\BuiltInTok}[1]{\textcolor[rgb]{0.00,0.23,0.31}{#1}}
\newcommand{\CharTok}[1]{\textcolor[rgb]{0.13,0.47,0.30}{#1}}
\newcommand{\CommentTok}[1]{\textcolor[rgb]{0.37,0.37,0.37}{#1}}
\newcommand{\CommentVarTok}[1]{\textcolor[rgb]{0.37,0.37,0.37}{\textit{#1}}}
\newcommand{\ConstantTok}[1]{\textcolor[rgb]{0.56,0.35,0.01}{#1}}
\newcommand{\ControlFlowTok}[1]{\textcolor[rgb]{0.00,0.23,0.31}{\textbf{#1}}}
\newcommand{\DataTypeTok}[1]{\textcolor[rgb]{0.68,0.00,0.00}{#1}}
\newcommand{\DecValTok}[1]{\textcolor[rgb]{0.68,0.00,0.00}{#1}}
\newcommand{\DocumentationTok}[1]{\textcolor[rgb]{0.37,0.37,0.37}{\textit{#1}}}
\newcommand{\ErrorTok}[1]{\textcolor[rgb]{0.68,0.00,0.00}{#1}}
\newcommand{\ExtensionTok}[1]{\textcolor[rgb]{0.00,0.23,0.31}{#1}}
\newcommand{\FloatTok}[1]{\textcolor[rgb]{0.68,0.00,0.00}{#1}}
\newcommand{\FunctionTok}[1]{\textcolor[rgb]{0.28,0.35,0.67}{#1}}
\newcommand{\ImportTok}[1]{\textcolor[rgb]{0.00,0.46,0.62}{#1}}
\newcommand{\InformationTok}[1]{\textcolor[rgb]{0.37,0.37,0.37}{#1}}
\newcommand{\KeywordTok}[1]{\textcolor[rgb]{0.00,0.23,0.31}{\textbf{#1}}}
\newcommand{\NormalTok}[1]{\textcolor[rgb]{0.00,0.23,0.31}{#1}}
\newcommand{\OperatorTok}[1]{\textcolor[rgb]{0.37,0.37,0.37}{#1}}
\newcommand{\OtherTok}[1]{\textcolor[rgb]{0.00,0.23,0.31}{#1}}
\newcommand{\PreprocessorTok}[1]{\textcolor[rgb]{0.68,0.00,0.00}{#1}}
\newcommand{\RegionMarkerTok}[1]{\textcolor[rgb]{0.00,0.23,0.31}{#1}}
\newcommand{\SpecialCharTok}[1]{\textcolor[rgb]{0.37,0.37,0.37}{#1}}
\newcommand{\SpecialStringTok}[1]{\textcolor[rgb]{0.13,0.47,0.30}{#1}}
\newcommand{\StringTok}[1]{\textcolor[rgb]{0.13,0.47,0.30}{#1}}
\newcommand{\VariableTok}[1]{\textcolor[rgb]{0.07,0.07,0.07}{#1}}
\newcommand{\VerbatimStringTok}[1]{\textcolor[rgb]{0.13,0.47,0.30}{#1}}
\newcommand{\WarningTok}[1]{\textcolor[rgb]{0.37,0.37,0.37}{\textit{#1}}}

\usepackage{longtable,booktabs,array}
\usepackage{calc} % for calculating minipage widths
% Correct order of tables after \paragraph or \subparagraph
\usepackage{etoolbox}
\makeatletter
\patchcmd\longtable{\par}{\if@noskipsec\mbox{}\fi\par}{}{}
\makeatother
% Allow footnotes in longtable head/foot
\IfFileExists{footnotehyper.sty}{\usepackage{footnotehyper}}{\usepackage{footnote}}
\makesavenoteenv{longtable}
\usepackage{graphicx}
\makeatletter
\newsavebox\pandoc@box
\newcommand*\pandocbounded[1]{% scales image to fit in text height/width
  \sbox\pandoc@box{#1}%
  \Gscale@div\@tempa{\textheight}{\dimexpr\ht\pandoc@box+\dp\pandoc@box\relax}%
  \Gscale@div\@tempb{\linewidth}{\wd\pandoc@box}%
  \ifdim\@tempb\p@<\@tempa\p@\let\@tempa\@tempb\fi% select the smaller of both
  \ifdim\@tempa\p@<\p@\scalebox{\@tempa}{\usebox\pandoc@box}%
  \else\usebox{\pandoc@box}%
  \fi%
}
% Set default figure placement to htbp
\def\fps@figure{htbp}
\makeatother


% definitions for citeproc citations
\NewDocumentCommand\citeproctext{}{}
\NewDocumentCommand\citeproc{mm}{%
  \begingroup\def\citeproctext{#2}\cite{#1}\endgroup}
\makeatletter
 % allow citations to break across lines
 \let\@cite@ofmt\@firstofone
 % avoid brackets around text for \cite:
 \def\@biblabel#1{}
 \def\@cite#1#2{{#1\if@tempswa , #2\fi}}
\makeatother
\newlength{\cslhangindent}
\setlength{\cslhangindent}{1.5em}
\newlength{\csllabelwidth}
\setlength{\csllabelwidth}{3em}
\newenvironment{CSLReferences}[2] % #1 hanging-indent, #2 entry-spacing
 {\begin{list}{}{%
  \setlength{\itemindent}{0pt}
  \setlength{\leftmargin}{0pt}
  \setlength{\parsep}{0pt}
  % turn on hanging indent if param 1 is 1
  \ifodd #1
   \setlength{\leftmargin}{\cslhangindent}
   \setlength{\itemindent}{-1\cslhangindent}
  \fi
  % set entry spacing
  \setlength{\itemsep}{#2\baselineskip}}}
 {\end{list}}
\usepackage{calc}
\newcommand{\CSLBlock}[1]{\hfill\break\parbox[t]{\linewidth}{\strut\ignorespaces#1\strut}}
\newcommand{\CSLLeftMargin}[1]{\parbox[t]{\csllabelwidth}{\strut#1\strut}}
\newcommand{\CSLRightInline}[1]{\parbox[t]{\linewidth - \csllabelwidth}{\strut#1\strut}}
\newcommand{\CSLIndent}[1]{\hspace{\cslhangindent}#1}



\setlength{\emergencystretch}{3em} % prevent overfull lines

\providecommand{\tightlist}{%
  \setlength{\itemsep}{0pt}\setlength{\parskip}{0pt}}



 


\usepackage{cogsci}
\usepackage{pslatex}
\setlength{\parskip}{0pt}
\setlength{\parindent}{1em}
\tolerance=9999
\emergencystretch=3em
\hyphenpenalty=1000
\exhyphenpenalty=1000
\makeatletter
\@ifpackageloaded{caption}{}{\usepackage{caption}}
\AtBeginDocument{%
\ifdefined\contentsname
  \renewcommand*\contentsname{Table of contents}
\else
  \newcommand\contentsname{Table of contents}
\fi
\ifdefined\listfigurename
  \renewcommand*\listfigurename{List of Figures}
\else
  \newcommand\listfigurename{List of Figures}
\fi
\ifdefined\listtablename
  \renewcommand*\listtablename{List of Tables}
\else
  \newcommand\listtablename{List of Tables}
\fi
\ifdefined\figurename
  \renewcommand*\figurename{Figure}
\else
  \newcommand\figurename{Figure}
\fi
\ifdefined\tablename
  \renewcommand*\tablename{Table}
\else
  \newcommand\tablename{Table}
\fi
}
\@ifpackageloaded{float}{}{\usepackage{float}}
\floatstyle{ruled}
\@ifundefined{c@chapter}{\newfloat{codelisting}{h}{lop}}{\newfloat{codelisting}{h}{lop}[chapter]}
\floatname{codelisting}{Listing}
\newcommand*\listoflistings{\listof{codelisting}{List of Listings}}
\makeatother
\makeatletter
\makeatother
\makeatletter
\@ifpackageloaded{caption}{}{\usepackage{caption}}
\@ifpackageloaded{subcaption}{}{\usepackage{subcaption}}
\makeatother
\usepackage{bookmark}
\IfFileExists{xurl.sty}{\usepackage{xurl}}{} % add URL line breaks if available
\urlstyle{same}
\hypersetup{
  pdftitle={Procedural Learning with Graded Entropy},
  colorlinks=true,
  linkcolor={blue},
  filecolor={Maroon},
  citecolor={Blue},
  urlcolor={Blue},
  pdfcreator={LaTeX via pandoc}}


\title{Procedural Learning with Graded Entropy}
\author{}
\date{}
\begin{document}
\author{{\bf Cherrie Chang}$^1$ \and {\bf Tianyi Li}$^2$ \and {\bf Joshua Hartshorne}$^1$ \\
  $^1$Department of Communication Sciences and Disorders, Mass General Hospital Institute of Health Professions, Charlestown, MA, USA \\
  $^2$Department of Psychology and Neuroscience, Boston College, Chestnut Hill, MA, USA \\
  cherriechang@gmail.com, lidzr@bc.edu, JKHartshorne@gmail.com}
\maketitle


\begin{Shaded}
\begin{Highlighting}[]
\FunctionTok{library}\NormalTok{(here)}
\FunctionTok{library}\NormalTok{(osfr)}
\FunctionTok{library}\NormalTok{(dplyr)}
\FunctionTok{library}\NormalTok{(tidyr)}
\FunctionTok{library}\NormalTok{(tidyverse)}
\FunctionTok{library}\NormalTok{(zoo)}
\FunctionTok{library}\NormalTok{(jsonlite)}
\FunctionTok{library}\NormalTok{(clipr)}
\FunctionTok{library}\NormalTok{(reticulate)}
\FunctionTok{library}\NormalTok{(lmerTest) }\CommentTok{\# show p value in lmer}
\FunctionTok{library}\NormalTok{(blme)}
\FunctionTok{library}\NormalTok{(purrr)}

\NormalTok{np }\OtherTok{\textless{}{-}} \FunctionTok{import}\NormalTok{(}\StringTok{"numpy"}\NormalTok{)}
\end{Highlighting}
\end{Shaded}

\section{Abstract}\label{abstract}

Your abstract text here. The abstract should be one paragraph. Following
the abstract should be keywords.

\textbf{Keywords:} procedural learning, implicit learning, statistical
learning

\section{Introductino}\label{introductino}

How many times have you typed ``form'' instead of ``from''? Or ended a
word with ``-tino'' instead of ``-tion'', perhaps even in an important,
embarrassingly glaring context? The typos we make are not random; they
reflect the statistical patterns in the language we use. Experienced
typists learn that some keystroke sequences are highly probable, causing
their fingers to automatically perform ones like ``for'' mistakenly, or
perform them so quickly they slip up, like in the case of ``-tino''. Yet
typists cannot consciously report these probabilities or articulate
their finger movements at the speed of execution---this knowledge is
expressed in the automatic motor performance of typing, not in conscious
awareness. This implicit form of learning, where statistical
regularities are unconsciously acquired and expressed as behavior rather
than declared knowledge, is called procedural learning (Squire (2004),
Nissen \& Bullemer (1987)).

Procedural learning is not limited to typing. It underlies skill
acquisition across many domains, from proprioceptive skills like riding
a bike to perceptual-cognitive skills like reading mirrored text
(Kassubek et al. (2001)). More recently, procedural learning has been
proposed to support increasingly diverse and sophisticated abilities,
notably grammar acquisition in language (Ullman \& Pierpont (2005),
Lammertink et al. (2017), Lum et al. (2014), Kidd \& Arciuli (2016)).
These proposals invite closer scrutiny of its scalability: can the
procedural learning system handle the large-scale, complex statistical
structures that characterize domains like grammar? Consider a suggestive
historical case: while QWERTY keyboards (26 letter keys) enabled fluent
touch typing through procedural learning, Chinese typewriters (trays of
2,000+ characters) never achieved comparable automaticity and were
eventually abandoned (for QWERTY keyboards, among others!) (Mullaney
(2017)). This raises a fundamental question: does procedural learning
fail beyond a certain level of statistical complexity and state space
size? As a first step toward answering these questions, we examine
whether procedural learning tracks fine-grained statistical
patterns---both the predictability of individual transitions and the
uncertainty of different states---and how it scales across state space
sizes from 4 to 8 positions.

The primary method for measuring procedural learning has been the Serial
Reaction Time Task (SRTT; Nissen \& Bullemer (1987)). In the SRTT,
participants respond to stimuli appearing in one of four positions on
screen using key presses. Unbeknownst to them, the sequence of positions
follows a deterministic pattern. Over time, reaction times (RT) decrease
for patterned trials relative to random trials, demonstrating procedural
learning of the key press sequence corresponding to the positional
sequence's deterministic pattern. While the SRTT provided early evidence
for procedural learning, it suffers from a critical confound:
participants often develop explicit awareness of the sequence, making it
difficult to isolate implicit procedural learning effects (Song et al.
(2008), Lustig (2022)). This has contributed to low test-retest
reliability of the task (West et al. (2018), Oliveira et al. (2023)).

The Alternating Serial Reaction Time Task (ASRT) addressed these issues
by introducing a probabilistic design. In ASRT, deterministic (d) and
random (r) trials alternate (e.g., d-r-d-r), resulting in positional
sequences with two levels of frequency (high vs.~low) (Howard Jr. \&
Howard (1997)). This probabilistic design minimizes explicit awareness
and dramatically improves reliability (Farkas et al. (2024), Oliveira et
al. (2023)). However, a key limitation of ASRT is that its bimodal
frequency distribution represents an overtly simplistic statistical
structure, lacking the graded frequency distributions found in
real-world domains like language. Finally, across both SRTT and ASRT,
experiment designs have mostly tested sequences with four possible
positions (e.g., Janacsek et al. (2012), Hedenius et al. (2011)),
despite real-world domains where procedural learning is implicated often
involving far larger state spaces. Whether procedural learning can track
graded statistical structures, and whether it scales to larger state
spaces, remains untested.

Here, we introduce a new paradigm that retains ASRT's probabilistic
design but introduces a graded statistical structure with a smooth
continuum of transition probabilities. We construct transition matrices
to generate positional sequences where each position has a different
entropy, ranging from deterministic (low entropy) to uniformly
distributed (high entropy). We also vary the number of positions (4, 5,
6, 7, 8) to test whether procedural learning scales with state space
size. This allows us to ask: (1) Are learners sensitive to fine-grained
differences in transition probabilities (surprisal) and state-level
uncertainty (entropy), beyond simple high vs.~low frequency contrasts?
(2) Do learning trajectories change as the number of positions
increases? Across 219 participants, we find evidence that procedural
learning is sensitive to graded statistical structure, and that this
sensitivity emerges early and strengthens over time. These results
suggest that procedural learning is more flexible than prior paradigms
can reveal, with implications for understanding how procedural learning
operates in naturalistic environments.

\subsection{Hypotheses}\label{hypotheses}

We make the following predictions: First, if participants are learning
the statistical structure and not merely practicing motor responses,
trial-level RT and accuracy will be predicted by surprisal-----the
negative log probability of the observed transition-----such that
higher-surprisal (less predictable) transitions elicit slower and less
accurate responses (surprisal effect). Second, previous positional
entropy-----the entropy of the transition distribution from the previous
position-----will modulate performance independently, with
higher-entropy (more uncertain) previous positions leading to slower and
less accurate responses (entropy effect). We further predict that
surprisal effects will be stronger following low-entropy positions,
where predictions are more confident and prediction errors more costly
(surprisal × entropy interaction). Third, these effects will be
attenuated as the number of positions increases, reflecting greater
difficulty in learning more complex statistical structures (state space
size effect).

\section{Method}\label{method}

\subsection{Participants}\label{participants}

\begin{Shaded}
\begin{Highlighting}[]
\NormalTok{df.demographics }\OtherTok{\textless{}{-}} \FunctionTok{list.files}\NormalTok{(}\FunctionTok{here}\NormalTok{(}\StringTok{"demographic{-}data/"}\NormalTok{), }\AttributeTok{pattern =} \StringTok{"*.csv"}\NormalTok{, }\AttributeTok{full.names =} \ConstantTok{TRUE}\NormalTok{) }\SpecialCharTok{\%\textgreater{}\%}
    \FunctionTok{lapply}\NormalTok{(}\ControlFlowTok{function}\NormalTok{(f) \{}
\NormalTok{        matrix\_size }\OtherTok{\textless{}{-}} \FunctionTok{gsub}\NormalTok{(}\StringTok{".*\_(}\SpecialCharTok{\textbackslash{}\textbackslash{}}\StringTok{dx}\SpecialCharTok{\textbackslash{}\textbackslash{}}\StringTok{d)}\SpecialCharTok{\textbackslash{}\textbackslash{}}\StringTok{.csv"}\NormalTok{, }\StringTok{"}\SpecialCharTok{\textbackslash{}\textbackslash{}}\StringTok{1"}\NormalTok{, }\FunctionTok{basename}\NormalTok{(f))}
        \FunctionTok{read.csv}\NormalTok{(f) }\SpecialCharTok{\%\textgreater{}\%}
            \FunctionTok{mutate}\NormalTok{(}\AttributeTok{Matrix.size =}\NormalTok{ matrix\_size)}
\NormalTok{    \}) }\SpecialCharTok{\%\textgreater{}\%}
    \FunctionTok{bind\_rows}\NormalTok{() }\SpecialCharTok{\%\textgreater{}\%}
    \FunctionTok{filter}\NormalTok{(Status }\SpecialCharTok{==} \StringTok{"APPROVED"}\NormalTok{)}

\NormalTok{overall\_N }\OtherTok{\textless{}{-}} \FunctionTok{nrow}\NormalTok{(df.demographics)}
\NormalTok{overall\_mean\_age }\OtherTok{\textless{}{-}} \FunctionTok{mean}\NormalTok{(}\FunctionTok{as.numeric}\NormalTok{(df.demographics}\SpecialCharTok{$}\NormalTok{Age), }\AttributeTok{na.rm =} \ConstantTok{TRUE}\NormalTok{)}
\NormalTok{overall\_median\_age }\OtherTok{\textless{}{-}} \FunctionTok{median}\NormalTok{(}\FunctionTok{as.numeric}\NormalTok{(df.demographics}\SpecialCharTok{$}\NormalTok{Age), }\AttributeTok{na.rm =} \ConstantTok{TRUE}\NormalTok{)}
\NormalTok{overall\_sd\_age }\OtherTok{\textless{}{-}} \FunctionTok{sd}\NormalTok{(}\FunctionTok{as.numeric}\NormalTok{(df.demographics}\SpecialCharTok{$}\NormalTok{Age), }\AttributeTok{na.rm =} \ConstantTok{TRUE}\NormalTok{)}
\NormalTok{overall\_age\_range }\OtherTok{\textless{}{-}} \FunctionTok{range}\NormalTok{(}\FunctionTok{as.numeric}\NormalTok{(df.demographics}\SpecialCharTok{$}\NormalTok{Age), }\AttributeTok{na.rm =} \ConstantTok{TRUE}\NormalTok{)}

\NormalTok{N\_nationalities }\OtherTok{\textless{}{-}} \FunctionTok{length}\NormalTok{(}\FunctionTok{unique}\NormalTok{(df.demographics}\SpecialCharTok{$}\NormalTok{Nationality))}
\NormalTok{N\_languages }\OtherTok{\textless{}{-}} \FunctionTok{length}\NormalTok{(}\FunctionTok{unique}\NormalTok{(df.demographics}\SpecialCharTok{$}\NormalTok{Language))}

\CommentTok{\# Gender distribution}
\NormalTok{gender\_counts }\OtherTok{\textless{}{-}} \FunctionTok{table}\NormalTok{(df.demographics}\SpecialCharTok{$}\NormalTok{Sex)}
\NormalTok{n\_female }\OtherTok{\textless{}{-}}\NormalTok{ gender\_counts[}\StringTok{"Female"}\NormalTok{]}
\NormalTok{n\_male }\OtherTok{\textless{}{-}}\NormalTok{ gender\_counts[}\StringTok{"Male"}\NormalTok{]}
\NormalTok{n\_other }\OtherTok{\textless{}{-}} \FunctionTok{sum}\NormalTok{(gender\_counts[}\SpecialCharTok{!}\FunctionTok{names}\NormalTok{(gender\_counts) }\SpecialCharTok{\%in\%} \FunctionTok{c}\NormalTok{(}\StringTok{"Female"}\NormalTok{, }\StringTok{"Male"}\NormalTok{)], }\AttributeTok{na.rm =} \ConstantTok{TRUE}\NormalTok{)}

\NormalTok{df.demographics.group\_summary }\OtherTok{\textless{}{-}}\NormalTok{ df.demographics }\SpecialCharTok{\%\textgreater{}\%}
    \FunctionTok{group\_by}\NormalTok{(Matrix.size) }\SpecialCharTok{\%\textgreater{}\%}
    \FunctionTok{summarise}\NormalTok{(}
        \AttributeTok{N =} \FunctionTok{n}\NormalTok{(),}
        \AttributeTok{mean\_age =} \FunctionTok{mean}\NormalTok{(}\FunctionTok{as.numeric}\NormalTok{(Age), }\AttributeTok{na.rm =} \ConstantTok{TRUE}\NormalTok{),}
        \AttributeTok{sd\_age =} \FunctionTok{sd}\NormalTok{(}\FunctionTok{as.numeric}\NormalTok{(Age), }\AttributeTok{na.rm =} \ConstantTok{TRUE}\NormalTok{),}
        \AttributeTok{min\_age =} \FunctionTok{min}\NormalTok{(}\FunctionTok{as.numeric}\NormalTok{(Age), }\AttributeTok{na.rm =} \ConstantTok{TRUE}\NormalTok{),}
        \AttributeTok{max\_age =} \FunctionTok{max}\NormalTok{(}\FunctionTok{as.numeric}\NormalTok{(Age), }\AttributeTok{na.rm =} \ConstantTok{TRUE}\NormalTok{)}
\NormalTok{    )}

\NormalTok{group\_N }\OtherTok{\textless{}{-}} \FunctionTok{first}\NormalTok{(df.demographics.group\_summary}\SpecialCharTok{$}\NormalTok{N)}
\NormalTok{N\_5x5 }\OtherTok{\textless{}{-}}\NormalTok{ df.demographics.group\_summary }\SpecialCharTok{\%\textgreater{}\%}
    \FunctionTok{filter}\NormalTok{(Matrix.size }\SpecialCharTok{==} \StringTok{"5x5"}\NormalTok{) }\SpecialCharTok{\%\textgreater{}\%}
    \FunctionTok{pull}\NormalTok{(N)}
\end{Highlighting}
\end{Shaded}

We recruited 251 participants (133 female, 118 male) to complete a
web-based experiment using the online crowd-sourcing platform Prolific,
screening for participants with an approval rating above 95\% from
previous studies and limiting participation to be on desktop or laptop
computers. Participants ranged in age from 18 to 75 years (\(\mu\)=35.6,
\(\sigma\)=11.9), representing 32 nationalities and 20 primary
languages. Participants were compensated at \$12.00 per hour, with
median completion time varying by condition (\textasciitilde20-40
minutes). All participants provided informed consent prior to the
experiment. The study was approved by the Institutional Review Board at
the Mass General Hospital Institute of Health Professions.

\subsection{Experiment Design}\label{experiment-design}

\begin{Shaded}
\begin{Highlighting}[]
\NormalTok{EXPERIMENT\_CONFIG }\OtherTok{\textless{}{-}} \FunctionTok{list}\NormalTok{(}
    \AttributeTok{osf\_id\_4x4 =} \StringTok{"Emfhw"}\NormalTok{,}
    \AttributeTok{osf\_id\_5x5 =} \StringTok{"Fraxt"}\NormalTok{,}
    \AttributeTok{osf\_id\_6x6 =} \StringTok{"jx48y"}\NormalTok{,}
    \AttributeTok{osf\_id\_7x7 =} \StringTok{"92jq7"}\NormalTok{,}
    \AttributeTok{osf\_id\_8x8 =} \StringTok{"Tx3h7"}\NormalTok{,}
    \AttributeTok{key\_mapping\_4pos =} \FunctionTok{c}\NormalTok{(}\StringTok{"d"}\NormalTok{, }\StringTok{"f"}\NormalTok{, }\StringTok{"j"}\NormalTok{, }\StringTok{"k"}\NormalTok{),}
    \AttributeTok{key\_mapping\_5pos =} \FunctionTok{c}\NormalTok{(}\StringTok{"s"}\NormalTok{, }\StringTok{"d"}\NormalTok{, }\StringTok{"f"}\NormalTok{, }\StringTok{"j"}\NormalTok{, }\StringTok{"k"}\NormalTok{),}
    \AttributeTok{key\_mapping\_6pos =} \FunctionTok{c}\NormalTok{(}\StringTok{"s"}\NormalTok{, }\StringTok{"d"}\NormalTok{, }\StringTok{"f"}\NormalTok{, }\StringTok{"j"}\NormalTok{, }\StringTok{"k"}\NormalTok{, }\StringTok{"l"}\NormalTok{),}
    \AttributeTok{key\_mapping\_7pos =} \FunctionTok{c}\NormalTok{(}\StringTok{"a"}\NormalTok{, }\StringTok{"s"}\NormalTok{, }\StringTok{"d"}\NormalTok{, }\StringTok{"f"}\NormalTok{, }\StringTok{"j"}\NormalTok{, }\StringTok{"k"}\NormalTok{, }\StringTok{"l"}\NormalTok{),}
    \AttributeTok{key\_mapping\_8pos =} \FunctionTok{c}\NormalTok{(}\StringTok{"a"}\NormalTok{, }\StringTok{"s"}\NormalTok{, }\StringTok{"d"}\NormalTok{, }\StringTok{"f"}\NormalTok{, }\StringTok{"j"}\NormalTok{, }\StringTok{"k"}\NormalTok{, }\StringTok{"l"}\NormalTok{, }\StringTok{";"}\NormalTok{),}
    \AttributeTok{n\_blocks =} \DecValTok{20}\NormalTok{,}
    \AttributeTok{rsi =} \DecValTok{120}\NormalTok{,}
    \AttributeTok{error\_feedback\_duration =} \DecValTok{200}\NormalTok{,}
    \AttributeTok{error\_tone\_duration =} \DecValTok{100}\NormalTok{,}
    \AttributeTok{correct\_feedback\_duration =} \DecValTok{200}\NormalTok{,}
    \AttributeTok{estimated\_trial\_duration =} \DecValTok{500}\NormalTok{,}
    \AttributeTok{accuracy\_threshold =} \FloatTok{0.65}\NormalTok{,}
    \AttributeTok{rt\_threshold =} \DecValTok{1000}
\NormalTok{)}
\end{Highlighting}
\end{Shaded}

\begin{table}[t]
\centering
\caption{Key mappings by number of positions}
\label{tab:key-mapping}

\begin{tabular*}{\columnwidth}{@{} c @{\extracolsep{\fill}} l @{}}
\toprule
\textit{n} positions & keys (left to right) \\
\midrule
4 & \{D, F, J, K\} \\
5 & \{S, D, F, J, K\} \\
6 & \{S, D, F, J, K, L\} \\
7 & \{A, S, D, F, J, K, L\} \\
8 & \{A, S, D, F, J, K, L, ;\} \\
\bottomrule
\end{tabular*}

\end{table}

\begin{figure}

\centering{

\includegraphics[width=1\linewidth,height=\textheight,keepaspectratio]{"../assets/example-trial-screen.png"}

}

\caption{\label{fig-example-trial-screen}An example trial screen in the
8-position condition of the experiment}

\end{figure}%

Like other serial reaction time task designs, our experiment tasks
participants to respond to a visual stimulus that appears in one of
several evenly-spaced positions on screen as quickly and accurately as
possible by pressing a corresponding key on the keyboard. In our
implementation, this visual stimulus is a mole garbed in a bright red
bib and matching sunglasses (Figure~\ref{fig-example-trial-screen}).
Participants were evenly divided into 5 groups of 50 per condition, with
the conditions differing in the number of positions (4, 5, 6, 7, 8) the
mole can appear in. (\textbf{tab:key-mapping?}) shows the
position-to-key mapping for each condition.

A participant is first given a text-based tutorial accompanied by an
animated demonstration on which key to press in response to each
position, then instructed to work through a set of practice trials where
each position is visited twice in random order. After the practice
section, the participant moves on to complete 20 blocks of trials, each
separated with a self-paced break. Each block consists of 10 trials per
position. This design choice results in longer blocks for conditions
with more positions, but ensures each position is visited an equal
number of times on average across conditions. In both practice and main
trials, participants are given feedback on correctness via a pop-up
short message (a checkmark vs.~``Try again!'' + an error tone), and
allowed to retry each failed trial until they respond correctly.
Following standard SRTT protocol, there is also a 120ms
response-stimulus interval (RSI) between trials (Nissen \& Bullemer
(1987), Janacsek et al. (2012), Howard et al. (2004)). We find during
piloting that allowing retries, combined with the 120ms RSI, prevents
participants from making compensatory errors, where they unintentionally
press the wrong key in the next trial in an attempt to correct their
current trial. We record response time, keyboard response and
correctness for each trial. After each block, participants are given
adaptive feedback based on their accuracy and speed. At the end of the
experiment, participants complete a brief questionnaire probing their
explicit awareness of any patterns in the mole's appearances (Table).

In addition to our dapper mole, a number of design choices were made to
facilitate participant understanding and engagement. During the practice
phase, each position on screen is labelled with its corresponding key
character (e.g.~``A'', ``S'', ``D'') to help familiarize participants
with the keyboard mappings. These key labels disappear in the main
trials to prevent explicitly encoding of the sequence of positions via
their character labels, but reappear during retry of failed trials. To
make the correspondence between the positions on screen and keyboard
keys as automatic and intuitive as possible, we chose to use
conventional QWERTY-based finger layouts for the key mappings across
conditions ((\textbf{tab:key-mapping?})). We found during piloting that
while this decision helped reduce cognitive load from learning novel
finger placements, the larger gap between keys ``F'' and ``J'' on the
keyboard introduced a discrepancy between the evenly-spaced on-screen
positions and the unevenly spaced keyboard keys. This made participants
more prone to making errors in the positions towards the middle of the
screen compared to those on the outer edges, especially when number of
positions increases {[}TODO: maybe run stats on pilot to find this
effect{]}; and was not mitigated by shifting either hand's placement to
close the extra gap, because the positions were still corresponding half
to one hand and half to the other. After further piloting testing
different visual aids, we settled on adding two hands visually to the
bottom of the screen, each finger aligned with its target position
(Figure). We received verbal feedback that this visual aid helped
participants better map the spatial layout of the on-screen positions to
their corresponding keys. Lastly, the visual design of the ``positions''
on screen mimicked blank 3D keyboard keys that light up in pink when the
mole appears on top, and look visibly ``pressed down'' when the
participant presses the corresponding keyboard key. This design choice
links the on-screen positions diegetically to the physical keyboard keys
to maximally reduce mental distance between the two. This is important
to enforce as much learning via implicit motor memory as possible,
rather than via visual-spatial memory of the on-screen positions.

\subsection{Transition Matrices}\label{transition-matrices}

\begin{Shaded}
\begin{Highlighting}[]
\CommentTok{\# Shannon entropy (bits) of a single probability vector (skip zeros)}
\NormalTok{row\_entropy }\OtherTok{\textless{}{-}} \ControlFlowTok{function}\NormalTok{(prob\_vec) \{}
\NormalTok{    p }\OtherTok{\textless{}{-}}\NormalTok{ prob\_vec[prob\_vec }\SpecialCharTok{\textgreater{}} \DecValTok{0}\NormalTok{]}
    \SpecialCharTok{{-}}\FunctionTok{sum}\NormalTok{(p }\SpecialCharTok{*} \FunctionTok{log2}\NormalTok{(p))}
\NormalTok{\}}

\CommentTok{\# Calculate positional entropy for each row in a transition matrix}
\NormalTok{positional\_entropy }\OtherTok{\textless{}{-}} \ControlFlowTok{function}\NormalTok{(mat) \{}
    \FunctionTok{apply}\NormalTok{(mat, }\AttributeTok{MARGIN =} \DecValTok{1}\NormalTok{, row\_entropy)}
\NormalTok{\}}

\NormalTok{load\_npy\_matrix }\OtherTok{\textless{}{-}} \ControlFlowTok{function}\NormalTok{(path) \{}
\NormalTok{  x }\OtherTok{\textless{}{-}}\NormalTok{ np}\SpecialCharTok{$}\FunctionTok{load}\NormalTok{(path)}
\NormalTok{  lst }\OtherTok{\textless{}{-}}\NormalTok{ x}\SpecialCharTok{$}\FunctionTok{tolist}\NormalTok{()          }\CommentTok{\# Python → nested lists}
  \FunctionTok{do.call}\NormalTok{(rbind, lst)        }\CommentTok{\# R{-}native matrix}
\NormalTok{\}}

\NormalTok{transition\_matrix\_4x4 }\OtherTok{\textless{}{-}} \FunctionTok{load\_npy\_matrix}\NormalTok{(}
  \FunctionTok{here}\NormalTok{(}\StringTok{"assets/transition{-}matrices/matrix\_4x4.npy"}\NormalTok{)}
\NormalTok{)}
\NormalTok{transition\_matrix\_5x5 }\OtherTok{\textless{}{-}} \FunctionTok{load\_npy\_matrix}\NormalTok{(}
  \FunctionTok{here}\NormalTok{(}\StringTok{"assets/transition{-}matrices/matrix\_5x5.npy"}\NormalTok{)}
\NormalTok{)}
\NormalTok{transition\_matrix\_6x6 }\OtherTok{\textless{}{-}} \FunctionTok{load\_npy\_matrix}\NormalTok{(}
  \FunctionTok{here}\NormalTok{(}\StringTok{"assets/transition{-}matrices/matrix\_6x6.npy"}\NormalTok{)}
\NormalTok{)}
\NormalTok{transition\_matrix\_7x7 }\OtherTok{\textless{}{-}} \FunctionTok{load\_npy\_matrix}\NormalTok{(}
  \FunctionTok{here}\NormalTok{(}\StringTok{"assets/transition{-}matrices/matrix\_7x7.npy"}\NormalTok{)}
\NormalTok{)}
\NormalTok{transition\_matrix\_8x8 }\OtherTok{\textless{}{-}} \FunctionTok{load\_npy\_matrix}\NormalTok{(}
  \FunctionTok{here}\NormalTok{(}\StringTok{"assets/transition{-}matrices/matrix\_8x8.npy"}\NormalTok{)}
\NormalTok{)}

\NormalTok{all\_original\_matrices }\OtherTok{\textless{}{-}} \FunctionTok{list}\NormalTok{(}
    \StringTok{"4x4"} \OtherTok{=}\NormalTok{ transition\_matrix\_4x4,}
    \StringTok{"5x5"} \OtherTok{=}\NormalTok{ transition\_matrix\_5x5,}
    \StringTok{"6x6"} \OtherTok{=}\NormalTok{ transition\_matrix\_6x6,}
    \StringTok{"7x7"} \OtherTok{=}\NormalTok{ transition\_matrix\_7x7,}
    \StringTok{"8x8"} \OtherTok{=}\NormalTok{ transition\_matrix\_8x8}
\NormalTok{)}

\NormalTok{all\_positional\_entropies }\OtherTok{\textless{}{-}} \FunctionTok{lapply}\NormalTok{(all\_original\_matrices, positional\_entropy)}
\NormalTok{all\_positional\_entropies\_df }\OtherTok{\textless{}{-}} \FunctionTok{lapply}\NormalTok{(}\FunctionTok{names}\NormalTok{(all\_positional\_entropies), }\ControlFlowTok{function}\NormalTok{(size) \{}
    \FunctionTok{data.frame}\NormalTok{(}
        \AttributeTok{Matrix.size =}\NormalTok{ size,}
        \AttributeTok{Position =} \DecValTok{1}\SpecialCharTok{:}\FunctionTok{length}\NormalTok{(all\_positional\_entropies[[size]]),}
        \AttributeTok{Entropy =}\NormalTok{ all\_positional\_entropies[[size]]}
\NormalTok{    )}
\NormalTok{\}) }\SpecialCharTok{\%\textgreater{}\%}
    \FunctionTok{bind\_rows}\NormalTok{()}

\NormalTok{pos\_1\_entropy }\OtherTok{\textless{}{-}}\NormalTok{ all\_positional\_entropies\_df }\SpecialCharTok{\%\textgreater{}\%}
    \FunctionTok{group\_by}\NormalTok{(Matrix.size) }\SpecialCharTok{\%\textgreater{}\%}
    \FunctionTok{filter}\NormalTok{(Position }\SpecialCharTok{==} \DecValTok{1}\NormalTok{) }\SpecialCharTok{\%\textgreater{}\%}
    \FunctionTok{ungroup}\NormalTok{() }\SpecialCharTok{\%\textgreater{}\%}
    \FunctionTok{summarize}\NormalTok{(}\AttributeTok{range =} \FunctionTok{range}\NormalTok{(Entropy), }\AttributeTok{mean =} \FunctionTok{mean}\NormalTok{(Entropy), }\AttributeTok{sd =} \FunctionTok{sd}\NormalTok{(Entropy))}

\NormalTok{pos\_n\_entropy }\OtherTok{\textless{}{-}}\NormalTok{ all\_positional\_entropies\_df }\SpecialCharTok{\%\textgreater{}\%}
    \FunctionTok{group\_by}\NormalTok{(Matrix.size) }\SpecialCharTok{\%\textgreater{}\%}
    \FunctionTok{filter}\NormalTok{(Position }\SpecialCharTok{==} \FunctionTok{max}\NormalTok{(Position)) }\SpecialCharTok{\%\textgreater{}\%}
    \FunctionTok{ungroup}\NormalTok{() }\SpecialCharTok{\%\textgreater{}\%}
    \FunctionTok{summarize}\NormalTok{(}\AttributeTok{range =} \FunctionTok{range}\NormalTok{(Entropy), }\AttributeTok{mean =} \FunctionTok{mean}\NormalTok{(Entropy), }\AttributeTok{sd =} \FunctionTok{sd}\NormalTok{(Entropy))}

\NormalTok{entropy\_gradient\_summary }\OtherTok{\textless{}{-}}\NormalTok{ all\_positional\_entropies\_df }\SpecialCharTok{\%\textgreater{}\%}
    \FunctionTok{group\_by}\NormalTok{(Matrix.size) }\SpecialCharTok{\%\textgreater{}\%}
    \FunctionTok{summarize}\NormalTok{(}\AttributeTok{entropy\_diffs =} \FunctionTok{list}\NormalTok{(}\FunctionTok{diff}\NormalTok{(Entropy))) }\SpecialCharTok{\%\textgreater{}\%}
    \FunctionTok{ungroup}\NormalTok{() }\SpecialCharTok{\%\textgreater{}\%}
    \FunctionTok{summarize}\NormalTok{(}
        \AttributeTok{range\_diff =} \FunctionTok{list}\NormalTok{(}\FunctionTok{range}\NormalTok{(}\FunctionTok{unlist}\NormalTok{(entropy\_diffs))),}
        \AttributeTok{mean\_diff =} \FunctionTok{mean}\NormalTok{(}\FunctionTok{unlist}\NormalTok{(entropy\_diffs)),}
        \AttributeTok{sd\_diff =} \FunctionTok{sd}\NormalTok{(}\FunctionTok{unlist}\NormalTok{(entropy\_diffs))}
\NormalTok{    )}
\end{Highlighting}
\end{Shaded}

Our experiment deviates from other serial reaction time tasks in how the
sequence of positions and its probabilistic structure is constructed. In
each run of the experiment, we use a first-order Markov process to
generate the sequence of positions the mole appears in. This process is
defined by a transition matrix, where each entry in the matrix defines
the probability of transitioning from one position to another. We
designed five different transition matrices of sizes 4x4, 5x5, 6x6, 7x7,
and 8x8 respectively corresponding to the five conditions in our
experiment. These ``original'' matrices were constructed to have graded
and increasing levels of entropy across rows, such that position 1 has
lowest entropy-having the most predictable next-position-transitions,
followed by the second, with the last position, position n, having
highest entropy. For example, in the 4x4 matrix, position 1 has a high
probability (\textasciitilde0.97) of transitioning to position 2, a low
probability (\textasciitilde0.03) of transitioning to position 1, and
never transitions to position 3 or 4; while position 4 has a more
uniform distribution of transition probabilities to all other positions
(to 1: \textasciitilde(0.39), to 2: \textasciitilde(0.03), to 3:
\textasciitilde(0.24), to 4: \textasciitilde(0.35)). This gives rise to
a gradient of entropy values across the positions, which results in a
richer, more varied distributed statistical structure underlying the
sequence of positions the mole appears in, allowing us to investigate
how the procedural learning system differentially picks up varying
levels of predictability in a given statistical structure.

Several design criteria/constraints were imposed on the construction of
these original transition matrices. First, these matrices had to be
doubly stochastic to ensure uniform visitation to each position over
time, preventing confounding effects from position frequency on
learning. Second, self-transitions (e.g.~position 1 to position 1)
should have near-zero probabilities to reduce data loss, as data from
repeated trials have been historically thrown out from analysis in
serial reaction time tasks to prevent confounding motor effects from
repeating the same keystroke {[}CITE{]}. Third, the graded entropy
levels across the positions should translate to graded probabilities
across bigrams as well, as having more fine-grained, distributed levels
of transition probabilities is our experiment's key distinction from
other SRTT tasks, such as the ASRT paradigm which only has 2 levels of
trigram probability. Fourth, the increase in entropy across positions
(i.e.~rows) should be significant between one another and as linear as
possible to ensure smooth a smooth gradient with meaningfully distinct
levels in predictability across positions.

Strictly satisfying all these constraints at once turned out to be not
only non-trivial, but mathematically impossible for our range of
n\_positions (i.e.~matrix sizes), so we wrote an optimization algorithm
that explores the candidate space for each matrix size, treating these
constraints as soft penalties to find a global minimum of the total
constraint violation, balancing trade-offs among the competing
constraints.

The five original matrices we constructed using this algorithm had these
properties: 1) near fully doubly stochastic---all row and column sums
sum to 1 (±0.0001); 2) approach a uniform stationary distribution after
7 timesteps where the stationary visitation for each state (position)
are within 0.001 of each other; 3) had a graded entropy structure across
states---each matrix had a positional entropy range of 0.061-0.181 bits
for position 1 (\(\mu\)=0.107, \(\sigma\)=0.048), and 1.695-2.835 bits
for position n (\(\mu\)=2.319, \(\sigma\)=0.451). Within a matrix each
position increases in entropy from the previous by 0.295-0.699 bits
(\(\mu\)=0.442, \(\sigma\)=0.085); and 4) avoided self loops as much as
possible---all of the matrices had \textless0.08 probabilities in the
left diagonal up until the last two positions. Within each matrix size
condition, all participants experience the same underlying entropy
gradient across positions, but the mapping between screen positions and
entropy levels is randomized through a double permutation procedure that
shuffles the original matrix (of that size) using the same random
permutation. This preserves the statistical properties of the original
matrix---transition probabilities and entropy values---but decouples
entropy levels from specific screen positions. The position sequence for
each participant is subsequently generated using their individual
shuffled transition matrix, so the sequence each participant experiences
is different both within and between matrix size conditions.

\section{Statistical Analysis}\label{statistical-analysis}

\begin{Shaded}
\begin{Highlighting}[]
\CommentTok{\# }\AlertTok{NOTE}\CommentTok{: Set eval=true only when you need to download data from OSF for the first time.}
\CommentTok{\# Once data is downloaded locally, keep eval=false to speed up rendering.}

\NormalTok{files\_4x4 }\OtherTok{\textless{}{-}} \FunctionTok{osf\_retrieve\_node}\NormalTok{(EXPERIMENT\_CONFIG}\SpecialCharTok{$}\NormalTok{osf\_id\_4x4) }\SpecialCharTok{\%\textgreater{}\%}
    \FunctionTok{osf\_ls\_files}\NormalTok{(}\AttributeTok{n\_max =} \ConstantTok{Inf}\NormalTok{) }\SpecialCharTok{\%\textgreater{}\%}
    \FunctionTok{osf\_download}\NormalTok{(}\AttributeTok{path =} \FunctionTok{here}\NormalTok{(}\StringTok{"data/4x4{-}data"}\NormalTok{), }\AttributeTok{conflicts =} \StringTok{"skip"}\NormalTok{)}

\NormalTok{files\_5x5 }\OtherTok{\textless{}{-}} \FunctionTok{osf\_retrieve\_node}\NormalTok{(EXPERIMENT\_CONFIG}\SpecialCharTok{$}\NormalTok{osf\_id\_5x5) }\SpecialCharTok{\%\textgreater{}\%}
    \FunctionTok{osf\_ls\_files}\NormalTok{(}\AttributeTok{n\_max =} \ConstantTok{Inf}\NormalTok{) }\SpecialCharTok{\%\textgreater{}\%}
    \FunctionTok{osf\_download}\NormalTok{(}\AttributeTok{path =} \FunctionTok{here}\NormalTok{(}\StringTok{"data/5x5{-}data"}\NormalTok{), }\AttributeTok{conflicts =} \StringTok{"skip"}\NormalTok{)}

\NormalTok{files\_6x6 }\OtherTok{\textless{}{-}} \FunctionTok{osf\_retrieve\_node}\NormalTok{(EXPERIMENT\_CONFIG}\SpecialCharTok{$}\NormalTok{osf\_id\_6x6) }\SpecialCharTok{\%\textgreater{}\%}
    \FunctionTok{osf\_ls\_files}\NormalTok{(}\AttributeTok{n\_max =} \ConstantTok{Inf}\NormalTok{) }\SpecialCharTok{\%\textgreater{}\%}
    \FunctionTok{osf\_download}\NormalTok{(}\AttributeTok{path =} \FunctionTok{here}\NormalTok{(}\StringTok{"data/6x6{-}data"}\NormalTok{), }\AttributeTok{conflicts =} \StringTok{"skip"}\NormalTok{)}

\NormalTok{files\_7x7 }\OtherTok{\textless{}{-}} \FunctionTok{osf\_retrieve\_node}\NormalTok{(EXPERIMENT\_CONFIG}\SpecialCharTok{$}\NormalTok{osf\_id\_7x7) }\SpecialCharTok{\%\textgreater{}\%}
    \FunctionTok{osf\_ls\_files}\NormalTok{(}\AttributeTok{n\_max =} \ConstantTok{Inf}\NormalTok{) }\SpecialCharTok{\%\textgreater{}\%}
    \FunctionTok{osf\_download}\NormalTok{(}\AttributeTok{path =} \FunctionTok{here}\NormalTok{(}\StringTok{"data/7x7{-}data"}\NormalTok{), }\AttributeTok{conflicts =} \StringTok{"skip"}\NormalTok{)}

\NormalTok{files\_8x8 }\OtherTok{\textless{}{-}} \FunctionTok{osf\_retrieve\_node}\NormalTok{(EXPERIMENT\_CONFIG}\SpecialCharTok{$}\NormalTok{osf\_id\_8x8) }\SpecialCharTok{\%\textgreater{}\%}
    \FunctionTok{osf\_ls\_files}\NormalTok{(}\AttributeTok{n\_max =} \ConstantTok{Inf}\NormalTok{) }\SpecialCharTok{\%\textgreater{}\%}
    \FunctionTok{osf\_download}\NormalTok{(}\AttributeTok{path =} \FunctionTok{here}\NormalTok{(}\StringTok{"data/8x8{-}data"}\NormalTok{), }\AttributeTok{conflicts =} \StringTok{"skip"}\NormalTok{)}
\end{Highlighting}
\end{Shaded}

\begin{Shaded}
\begin{Highlighting}[]
\CommentTok{\# But there is some bug in osfr that causes one file name to be duplicated and one file name to be skipped, so alternatively just download manually from OSF and put in data/ folder:}
\NormalTok{files\_4x4 }\OtherTok{\textless{}{-}} \FunctionTok{list.files}\NormalTok{(}\StringTok{"data/4x4{-}data"}\NormalTok{)}
\NormalTok{files\_5x5 }\OtherTok{\textless{}{-}} \FunctionTok{list.files}\NormalTok{(}\StringTok{"data/5x5{-}data"}\NormalTok{)}
\NormalTok{files\_6x6 }\OtherTok{\textless{}{-}} \FunctionTok{list.files}\NormalTok{(}\StringTok{"data/6x6{-}data"}\NormalTok{)}
\NormalTok{files\_7x7 }\OtherTok{\textless{}{-}} \FunctionTok{list.files}\NormalTok{(}\StringTok{"data/7x7{-}data"}\NormalTok{)}
\NormalTok{files\_8x8 }\OtherTok{\textless{}{-}} \FunctionTok{list.files}\NormalTok{(}\StringTok{"data/8x8{-}data"}\NormalTok{)}
\end{Highlighting}
\end{Shaded}

\begin{Shaded}
\begin{Highlighting}[]
\CommentTok{\# Cache df.raw to speed up rendering}
\NormalTok{df\_raw\_cache }\OtherTok{\textless{}{-}} \FunctionTok{here}\NormalTok{(}\StringTok{"data/processed/df\_raw.rds"}\NormalTok{)}

\ControlFlowTok{if}\NormalTok{ (}\FunctionTok{file.exists}\NormalTok{(df\_raw\_cache)) \{}
\NormalTok{    df.raw }\OtherTok{\textless{}{-}} \FunctionTok{readRDS}\NormalTok{(df\_raw\_cache)}
\NormalTok{\} }\ControlFlowTok{else}\NormalTok{ \{}
\NormalTok{    df.raw }\OtherTok{\textless{}{-}} \FunctionTok{list.files}\NormalTok{(}\FunctionTok{here}\NormalTok{(}\StringTok{"data/"}\NormalTok{), }\AttributeTok{pattern =} \StringTok{"*.csv"}\NormalTok{, }\AttributeTok{full.names =} \ConstantTok{TRUE}\NormalTok{, }\AttributeTok{recursive =} \ConstantTok{TRUE}\NormalTok{) }\SpecialCharTok{\%\textgreater{}\%}
\NormalTok{        .[}\SpecialCharTok{!}\FunctionTok{grepl}\NormalTok{(}\StringTok{"pilot{-}data"}\NormalTok{, .)] }\SpecialCharTok{\%\textgreater{}\%}
        \FunctionTok{lapply}\NormalTok{(read.csv) }\SpecialCharTok{\%\textgreater{}\%}
        \FunctionTok{bind\_rows}\NormalTok{()}

    \CommentTok{\# Create processed directory if it doesn\textquotesingle{}t exist}
    \FunctionTok{dir.create}\NormalTok{(}\FunctionTok{here}\NormalTok{(}\StringTok{"data/processed"}\NormalTok{), }\AttributeTok{showWarnings =} \ConstantTok{FALSE}\NormalTok{, }\AttributeTok{recursive =} \ConstantTok{TRUE}\NormalTok{)}
    \FunctionTok{saveRDS}\NormalTok{(df.raw, df\_raw\_cache)}
\NormalTok{\}}
\end{Highlighting}
\end{Shaded}

\begin{Shaded}
\begin{Highlighting}[]
\NormalTok{df }\OtherTok{\textless{}{-}} \FunctionTok{subset}\NormalTok{(df.raw, }\AttributeTok{select =} \FunctionTok{c}\NormalTok{(trial\_index, subject\_id, matrix\_size, trials\_per\_block, total\_trials, practice\_trials, transition\_matrix, conditional\_entropies, phase, block, experiment\_trial\_type, trial\_in\_block, overall\_trial, position, correct\_key, response, correct, rt, conditional\_entropy, surprisal, questionnaire\_item))}
\end{Highlighting}
\end{Shaded}

\begin{Shaded}
\begin{Highlighting}[]
\CommentTok{\# Convert NA values}
\NormalTok{df }\OtherTok{\textless{}{-}}\NormalTok{ df }\SpecialCharTok{\%\textgreater{}\%}
    \FunctionTok{mutate}\NormalTok{(}\FunctionTok{across}\NormalTok{(}\FunctionTok{where}\NormalTok{(is.character), }\SpecialCharTok{\textasciitilde{}}\NormalTok{ \{}
\NormalTok{        x }\OtherTok{\textless{}{-}}\NormalTok{ .}
\NormalTok{        x[x }\SpecialCharTok{\%in\%} \FunctionTok{c}\NormalTok{(}\StringTok{"NA"}\NormalTok{, }\StringTok{"na"}\NormalTok{, }\StringTok{"null"}\NormalTok{, }\StringTok{"NULL"}\NormalTok{, }\StringTok{""}\NormalTok{)] }\OtherTok{\textless{}{-}} \ConstantTok{NA}
\NormalTok{        x}
\NormalTok{    \}))}

\CommentTok{\# Convert boolean values}
\NormalTok{df }\OtherTok{\textless{}{-}}\NormalTok{ df }\SpecialCharTok{\%\textgreater{}\%}
    \FunctionTok{mutate}\NormalTok{(}\FunctionTok{across}\NormalTok{(}\FunctionTok{where}\NormalTok{(is.character), }\SpecialCharTok{\textasciitilde{}}\NormalTok{ \{}
\NormalTok{        x }\OtherTok{\textless{}{-}} \FunctionTok{tolower}\NormalTok{(.)}
        \ControlFlowTok{if}\NormalTok{ (}\FunctionTok{all}\NormalTok{(x }\SpecialCharTok{\%in\%} \FunctionTok{c}\NormalTok{(}\StringTok{"true"}\NormalTok{, }\StringTok{"false"}\NormalTok{, }\ConstantTok{NA}\NormalTok{))) \{}
            \FunctionTok{as.logical}\NormalTok{(x)}
\NormalTok{        \} }\ControlFlowTok{else}\NormalTok{ \{}
\NormalTok{            .}
\NormalTok{        \}}
\NormalTok{    \}))}

\CommentTok{\# Convert numeric values}
\NormalTok{df }\OtherTok{\textless{}{-}}\NormalTok{ df }\SpecialCharTok{\%\textgreater{}\%}
    \FunctionTok{mutate}\NormalTok{(}\FunctionTok{across}\NormalTok{(}\FunctionTok{where}\NormalTok{(is.character), }\SpecialCharTok{\textasciitilde{}}\NormalTok{ \{}
\NormalTok{        x }\OtherTok{\textless{}{-}}\NormalTok{ .}
\NormalTok{        is\_num }\OtherTok{\textless{}{-}} \FunctionTok{suppressWarnings}\NormalTok{(}\SpecialCharTok{!}\FunctionTok{is.na}\NormalTok{(}\FunctionTok{as.numeric}\NormalTok{(x)) }\SpecialCharTok{|} \FunctionTok{is.na}\NormalTok{(x))}
        \ControlFlowTok{if}\NormalTok{ (}\FunctionTok{all}\NormalTok{(is\_num)) }\FunctionTok{as.numeric}\NormalTok{(x) }\ControlFlowTok{else}\NormalTok{ x}
\NormalTok{    \}))}
\end{Highlighting}
\end{Shaded}

\begin{Shaded}
\begin{Highlighting}[]
\NormalTok{df }\OtherTok{\textless{}{-}}\NormalTok{ df }\SpecialCharTok{\%\textgreater{}\%}
    \FunctionTok{mutate}\NormalTok{(}
        \AttributeTok{matrix\_size\_c =}\NormalTok{ matrix\_size }\SpecialCharTok{{-}}\NormalTok{ ((}\FunctionTok{max}\NormalTok{(df}\SpecialCharTok{$}\NormalTok{matrix\_size, }\AttributeTok{na.rm =} \ConstantTok{TRUE}\NormalTok{) }\SpecialCharTok{+} \FunctionTok{min}\NormalTok{(df}\SpecialCharTok{$}\NormalTok{matrix\_size, }\AttributeTok{na.rm =} \ConstantTok{TRUE}\NormalTok{)) }\SpecialCharTok{/} \DecValTok{2}\NormalTok{),}
        \AttributeTok{block\_c =}\NormalTok{ block }\SpecialCharTok{{-}}\NormalTok{ ((}\FunctionTok{max}\NormalTok{(df}\SpecialCharTok{$}\NormalTok{block, }\AttributeTok{na.rm =} \ConstantTok{TRUE}\NormalTok{) }\SpecialCharTok{+} \FunctionTok{min}\NormalTok{(df}\SpecialCharTok{$}\NormalTok{block, }\AttributeTok{na.rm =} \ConstantTok{TRUE}\NormalTok{)) }\SpecialCharTok{/} \DecValTok{2}\NormalTok{)}
\NormalTok{    )}
\end{Highlighting}
\end{Shaded}

\begin{Shaded}
\begin{Highlighting}[]
\NormalTok{df }\OtherTok{\textless{}{-}}\NormalTok{ df }\SpecialCharTok{\%\textgreater{}\%}
    \FunctionTok{group\_by}\NormalTok{(subject\_id) }\SpecialCharTok{\%\textgreater{}\%}
    \FunctionTok{arrange}\NormalTok{(subject\_id, }\FunctionTok{row\_number}\NormalTok{()) }\SpecialCharTok{\%\textgreater{}\%}
    \FunctionTok{mutate}\NormalTok{(}
        \AttributeTok{overall\_trial =} \FunctionTok{if\_else}\NormalTok{(}
\NormalTok{            experiment\_trial\_type }\SpecialCharTok{\%in\%} \FunctionTok{c}\NormalTok{(}\StringTok{"feedback"}\NormalTok{, }\StringTok{"rsi"}\NormalTok{),}
\NormalTok{            zoo}\SpecialCharTok{::}\FunctionTok{na.locf}\NormalTok{(overall\_trial, }\AttributeTok{na.rm =} \ConstantTok{FALSE}\NormalTok{), }\CommentTok{\# Fill down}
\NormalTok{            overall\_trial }\CommentTok{\# Keep as{-}is}
\NormalTok{        ),}
        \AttributeTok{trial\_in\_block =} \FunctionTok{if\_else}\NormalTok{(}
\NormalTok{            experiment\_trial\_type }\SpecialCharTok{\%in\%} \FunctionTok{c}\NormalTok{(}\StringTok{"feedback"}\NormalTok{, }\StringTok{"rsi"}\NormalTok{),}
\NormalTok{            zoo}\SpecialCharTok{::}\FunctionTok{na.locf}\NormalTok{(trial\_in\_block, }\AttributeTok{na.rm =} \ConstantTok{FALSE}\NormalTok{),}
\NormalTok{            trial\_in\_block}
\NormalTok{        ),}
        \AttributeTok{block =} \FunctionTok{if\_else}\NormalTok{(}
\NormalTok{            experiment\_trial\_type }\SpecialCharTok{\%in\%} \FunctionTok{c}\NormalTok{(}\StringTok{"feedback"}\NormalTok{, }\StringTok{"rsi"}\NormalTok{),}
\NormalTok{            zoo}\SpecialCharTok{::}\FunctionTok{na.locf}\NormalTok{(block, }\AttributeTok{na.rm =} \ConstantTok{FALSE}\NormalTok{),}
\NormalTok{            block}
\NormalTok{        ),}
        \AttributeTok{block\_c =} \FunctionTok{if\_else}\NormalTok{(}
\NormalTok{            experiment\_trial\_type }\SpecialCharTok{\%in\%} \FunctionTok{c}\NormalTok{(}\StringTok{"feedback"}\NormalTok{, }\StringTok{"rsi"}\NormalTok{),}
\NormalTok{            zoo}\SpecialCharTok{::}\FunctionTok{na.locf}\NormalTok{(block\_c, }\AttributeTok{na.rm =} \ConstantTok{FALSE}\NormalTok{),}
\NormalTok{            block\_c}
\NormalTok{        )}
\NormalTok{    ) }\SpecialCharTok{\%\textgreater{}\%}
    \FunctionTok{ungroup}\NormalTok{()}
\end{Highlighting}
\end{Shaded}

\begin{Shaded}
\begin{Highlighting}[]
\NormalTok{unique\_subjects }\OtherTok{\textless{}{-}} \FunctionTok{unique}\NormalTok{(df}\SpecialCharTok{$}\NormalTok{subject\_id)}

\CommentTok{\# Per{-}subject constants grouped by matrix\_size}
\NormalTok{df }\SpecialCharTok{\%\textgreater{}\%}
    \FunctionTok{distinct}\NormalTok{(matrix\_size, subject\_id, total\_trials, practice\_trials, transition\_matrix, conditional\_entropies) }\SpecialCharTok{\%\textgreater{}\%}
    \FunctionTok{group\_by}\NormalTok{(matrix\_size) }\SpecialCharTok{\%\textgreater{}\%}
    \FunctionTok{summarize}\NormalTok{(}
        \AttributeTok{n\_subjects =} \FunctionTok{n\_distinct}\NormalTok{(subject\_id),}
        \AttributeTok{total\_trials =} \FunctionTok{unique}\NormalTok{(total\_trials),}
        \AttributeTok{practice\_trials =} \FunctionTok{unique}\NormalTok{(practice\_trials),}
        \AttributeTok{transition\_matrix =} \FunctionTok{list}\NormalTok{(}\FunctionTok{unique}\NormalTok{(transition\_matrix)),}
        \AttributeTok{conditional\_entropies =} \FunctionTok{list}\NormalTok{(}\FunctionTok{unique}\NormalTok{(conditional\_entropies))}
\NormalTok{    )}

\CommentTok{\# Unique position sequences per matrix\_size}
\NormalTok{df }\SpecialCharTok{\%\textgreater{}\%}
    \FunctionTok{group\_by}\NormalTok{(matrix\_size, subject\_id) }\SpecialCharTok{\%\textgreater{}\%}
    \FunctionTok{summarize}\NormalTok{(}
        \AttributeTok{sequence =} \FunctionTok{paste}\NormalTok{(position, }\AttributeTok{collapse =} \StringTok{","}\NormalTok{),}
        \AttributeTok{.groups =} \StringTok{"drop"}
\NormalTok{    ) }\SpecialCharTok{\%\textgreater{}\%}
    \FunctionTok{group\_by}\NormalTok{(matrix\_size) }\SpecialCharTok{\%\textgreater{}\%}
    \FunctionTok{summarize}\NormalTok{(}
        \AttributeTok{n\_subjects =} \FunctionTok{n\_distinct}\NormalTok{(subject\_id),}
        \AttributeTok{n\_unique\_sequences =} \FunctionTok{n\_distinct}\NormalTok{(sequence)}
\NormalTok{    )}

\NormalTok{df }\OtherTok{\textless{}{-}}\NormalTok{ df[, }\SpecialCharTok{!}\NormalTok{(}\FunctionTok{names}\NormalTok{(df) }\SpecialCharTok{\%in\%} \FunctionTok{c}\NormalTok{(}\StringTok{"transition\_matrix"}\NormalTok{, }\StringTok{"conditional\_entropies"}\NormalTok{))]}

\CommentTok{\# Shows only 1 unique transition matrix and conditional entropies for conditions 6x6, 7x7 and 8x8 because of an error in data collection for those conditions after removing row by row log of the transition matrix and conditional entropies from the data files to save space. However, the code to shuffle those matrices and corresponding conditional entropies are still the same, so we believe participants are still each seeing a shuffled matrix and a different sequence in these conditions.}
\end{Highlighting}
\end{Shaded}

\begin{Shaded}
\begin{Highlighting}[]
\NormalTok{df.practice }\OtherTok{\textless{}{-}}\NormalTok{ df[df}\SpecialCharTok{$}\NormalTok{phase }\SpecialCharTok{==} \StringTok{"practice"}\NormalTok{, ]}
\NormalTok{df.questionnaire }\OtherTok{\textless{}{-}}\NormalTok{ df[df}\SpecialCharTok{$}\NormalTok{phase }\SpecialCharTok{==} \StringTok{"questionnaire"}\NormalTok{, ]}
\NormalTok{df }\OtherTok{\textless{}{-}}\NormalTok{ df[df}\SpecialCharTok{$}\NormalTok{phase }\SpecialCharTok{==} \StringTok{"main"}\NormalTok{, ]}
\end{Highlighting}
\end{Shaded}

\begin{Shaded}
\begin{Highlighting}[]
\CommentTok{\# Exclude subjects that}
\CommentTok{\# 1. Made too many mistakes (more than 15\%)}
\CommentTok{\# 2. Have too long response time (look at mean, median and spread)}
\CommentTok{\# {-}\textgreater{} exclude if too slow{-}{-}the subject\textquotesingle{}s median\_rt \textgreater{} 1000ms}
\CommentTok{\# {-}\textgreater{} or if too noisy{-}{-}more than 20\% of their trials are 3*mad higher than the subject\textquotesingle{}s median\_rt}

\NormalTok{accuracy\_excluded\_subjects }\OtherTok{\textless{}{-}}\NormalTok{ df }\SpecialCharTok{\%\textgreater{}\%}
    \FunctionTok{filter}\NormalTok{(}\SpecialCharTok{!}\FunctionTok{is.na}\NormalTok{(correct)) }\SpecialCharTok{\%\textgreater{}\%}
    \FunctionTok{group\_by}\NormalTok{(subject\_id) }\SpecialCharTok{\%\textgreater{}\%}
    \FunctionTok{summarize}\NormalTok{(}\AttributeTok{error\_rate =} \FunctionTok{mean}\NormalTok{(}\SpecialCharTok{!}\NormalTok{correct)) }\SpecialCharTok{\%\textgreater{}\%}
    \FunctionTok{filter}\NormalTok{(error\_rate }\SpecialCharTok{\textgreater{}} \FloatTok{0.10}\NormalTok{) }\SpecialCharTok{\%\textgreater{}\%}
    \FunctionTok{pull}\NormalTok{(subject\_id)}

\NormalTok{rt\_excluded\_subjects }\OtherTok{\textless{}{-}}\NormalTok{ df }\SpecialCharTok{\%\textgreater{}\%}
    \FunctionTok{filter}\NormalTok{(}
\NormalTok{        (experiment\_trial\_type }\SpecialCharTok{==} \StringTok{"stimulus"} \SpecialCharTok{|}\NormalTok{ experiment\_trial\_type }\SpecialCharTok{==} \StringTok{"retry"}\NormalTok{) }\SpecialCharTok{\&}
            \SpecialCharTok{!}\FunctionTok{is.na}\NormalTok{(rt)}
\NormalTok{    ) }\SpecialCharTok{\%\textgreater{}\%}
    \FunctionTok{group\_by}\NormalTok{(subject\_id) }\SpecialCharTok{\%\textgreater{}\%}
    \FunctionTok{summarize}\NormalTok{(}
        \AttributeTok{median\_rt =} \FunctionTok{median}\NormalTok{(rt),}
        \AttributeTok{outlier\_prop =} \FunctionTok{mean}\NormalTok{(rt }\SpecialCharTok{\textgreater{}} \FunctionTok{median}\NormalTok{(rt) }\SpecialCharTok{+} \DecValTok{3} \SpecialCharTok{*} \FunctionTok{mad}\NormalTok{(rt))}
\NormalTok{    ) }\SpecialCharTok{\%\textgreater{}\%}
    \FunctionTok{filter}\NormalTok{(median\_rt }\SpecialCharTok{\textgreater{}} \DecValTok{1000} \SpecialCharTok{|}\NormalTok{ outlier\_prop }\SpecialCharTok{\textgreater{}} \FloatTok{0.2}\NormalTok{) }\SpecialCharTok{\%\textgreater{}\%}
    \FunctionTok{pull}\NormalTok{(subject\_id)}

\NormalTok{df }\OtherTok{\textless{}{-}}\NormalTok{ df }\SpecialCharTok{\%\textgreater{}\%}
    \FunctionTok{filter}\NormalTok{(}\SpecialCharTok{!}\NormalTok{subject\_id }\SpecialCharTok{\%in\%}\NormalTok{ accuracy\_excluded\_subjects }\SpecialCharTok{\&} \SpecialCharTok{!}\NormalTok{subject\_id }\SpecialCharTok{\%in\%}\NormalTok{ rt\_excluded\_subjects)}
\end{Highlighting}
\end{Shaded}

\begin{Shaded}
\begin{Highlighting}[]
\NormalTok{df }\OtherTok{\textless{}{-}}\NormalTok{ df }\SpecialCharTok{\%\textgreater{}\%}
    \FunctionTok{filter}\NormalTok{(experiment\_trial\_type }\SpecialCharTok{==} \StringTok{"stimulus"} \SpecialCharTok{\&} \SpecialCharTok{!}\FunctionTok{is.na}\NormalTok{(rt) }\SpecialCharTok{\&}\NormalTok{ rt }\SpecialCharTok{!=} \DecValTok{0}\NormalTok{)}
\end{Highlighting}
\end{Shaded}

\begin{Shaded}
\begin{Highlighting}[]
\NormalTok{df }\OtherTok{\textless{}{-}}\NormalTok{ df }\SpecialCharTok{\%\textgreater{}\%}
    \FunctionTok{group\_by}\NormalTok{(block, subject\_id) }\SpecialCharTok{\%\textgreater{}\%}
    \FunctionTok{slice}\NormalTok{(}\DecValTok{2}\SpecialCharTok{:}\FunctionTok{n}\NormalTok{())}
\end{Highlighting}
\end{Shaded}

\begin{Shaded}
\begin{Highlighting}[]
\CommentTok{\# Cache fully processed df}
\NormalTok{df\_processed\_cache }\OtherTok{\textless{}{-}} \FunctionTok{here}\NormalTok{(}\StringTok{"data/processed/df\_processed.rds"}\NormalTok{)}

\ControlFlowTok{if}\NormalTok{ (}\FunctionTok{file.exists}\NormalTok{(df\_processed\_cache)) \{}
\NormalTok{    df }\OtherTok{\textless{}{-}} \FunctionTok{readRDS}\NormalTok{(df\_processed\_cache)}
\NormalTok{\} }\ControlFlowTok{else}\NormalTok{ \{}
\NormalTok{    df }\OtherTok{\textless{}{-}}\NormalTok{ df }\SpecialCharTok{\%\textgreater{}\%}
        \FunctionTok{mutate}\NormalTok{(}\AttributeTok{log\_rt =} \FunctionTok{log}\NormalTok{(rt)) }\SpecialCharTok{\%\textgreater{}\%}
        \FunctionTok{mutate}\NormalTok{(}\AttributeTok{is\_repetition =}\NormalTok{ (position }\SpecialCharTok{==} \FunctionTok{lag}\NormalTok{(position, }\DecValTok{1}\NormalTok{))) }\SpecialCharTok{\%\textgreater{}\%}
        \FunctionTok{group\_by}\NormalTok{(subject\_id, block) }\SpecialCharTok{\%\textgreater{}\%}
        \FunctionTok{mutate}\NormalTok{(}\AttributeTok{previous\_entropy =} \FunctionTok{lag}\NormalTok{(conditional\_entropy, }\DecValTok{1}\NormalTok{)) }\SpecialCharTok{\%\textgreater{}\%}
        \FunctionTok{ungroup}\NormalTok{() }\SpecialCharTok{\%\textgreater{}\%}
        \FunctionTok{group\_by}\NormalTok{(matrix\_size) }\SpecialCharTok{\%\textgreater{}\%}
        \FunctionTok{mutate}\NormalTok{(}\AttributeTok{prev\_entropy\_c =}\NormalTok{ previous\_entropy }\SpecialCharTok{{-}} \FunctionTok{mean}\NormalTok{(previous\_entropy, }\AttributeTok{na.rm =} \ConstantTok{TRUE}\NormalTok{)) }\SpecialCharTok{\%\textgreater{}\%}
        \FunctionTok{mutate}\NormalTok{(}\AttributeTok{surprisal\_c =}\NormalTok{ surprisal }\SpecialCharTok{{-}} \FunctionTok{mean}\NormalTok{(surprisal, }\AttributeTok{na.rm =} \ConstantTok{TRUE}\NormalTok{)) }\SpecialCharTok{\%\textgreater{}\%}
        \FunctionTok{ungroup}\NormalTok{()}

    \FunctionTok{saveRDS}\NormalTok{(df, df\_processed\_cache)}
\NormalTok{\}}
\end{Highlighting}
\end{Shaded}

\begin{Shaded}
\begin{Highlighting}[]
\CommentTok{\# Cache correct trials df}
\NormalTok{df\_correct\_cache }\OtherTok{\textless{}{-}} \FunctionTok{here}\NormalTok{(}\StringTok{"data/processed/df\_correct.rds"}\NormalTok{)}

\ControlFlowTok{if}\NormalTok{ (}\FunctionTok{file.exists}\NormalTok{(df\_correct\_cache)) \{}
\NormalTok{    df.correct }\OtherTok{\textless{}{-}} \FunctionTok{readRDS}\NormalTok{(df\_correct\_cache)}
\NormalTok{\} }\ControlFlowTok{else}\NormalTok{ \{}
\NormalTok{    df.correct }\OtherTok{\textless{}{-}}\NormalTok{ df }\SpecialCharTok{\%\textgreater{}\%}
        \FunctionTok{filter}\NormalTok{(correct)}

    \FunctionTok{saveRDS}\NormalTok{(df.correct, df\_correct\_cache)}
\NormalTok{\}}
\end{Highlighting}
\end{Shaded}

\subsection{Variables and Exclusion
Criteria}\label{variables-and-exclusion-criteria}

The main independent variables are surprisal and previous positional
entropy. Surprisal is defined as the negative log probability of the
actual next-position given the current position {[}TODO: math{]},
derived from the transition matrix used to generate the sequence for
that participant. Previous positional entropy is defined as the Shannon
entropy of the transition probabilities from the previous position, also
derived from the transition matrix {[}TODO: math{]}. Both surprisal and
previous positional entropy are mean-centered within each matrix size
condition to facilitate interpretation of main effects and interactions
in subsequent models. We also create a binary variable indicating
whether the current position is a repetition of the previous position
(i.e.~self-transition). This variable is included as a covariate in
subsequent models to control for potential motor effects from repeating
the same keystroke.

The main dependent variables are response time (RT) and accuracy
(proportion of correct responses) on non-practice/main trials. Incorrect
trials, trials with RT \textgreater1000ms, and trials with RTs greater
than 3 median absolute deviations above the participant's median RT
(outliers) are excluded. On the participant level, participants with
overall accuracy below 85\%, median RT \textgreater1000ms, or with
outliers making up more than 30\% of total trials are excluded from
analysis. We also discard data from the first trial of each block to
account for blocks being implicitly treated as ``restarts'' and not
following an existing probability distribution by participants. RTs are
log-transformed to reduce skewness.

\subsection{Modeling Approach}\label{modeling-approach}

We use linear mixed effects regression (LMER) to model log-transformed
RTs and generalized linear mixed effects regression (GLMER) with a
binomial link function to model accuracy. Models are fit using the
\texttt{lmerTest} package in R. We first determine the optimal random
and fixed effects structure using the final block of trials in each
condition. We test candidate structrues by fitting them to each of the
five matrix size conditions separately and selecting the structure with
the lowest summed AIC across all conditions. Using this selected
structure, we then fit separate models to each individual block within
each matrix size condition to examine learning trajectories. Finally, we
extract the effect size estimates from each model and test whether they
are moderated by matrix size and block using linear regression without
random effects, as the data are aggregated coefficients.

\textbf{Random effects selection.} Models are initially fit with the
maximal random effects structure supported by the design: random
intercepts for participant and position, and random slopes by
participant for surprisal, previous positional entropy, their
interaction, and whether the current position is a repetition of the
previous position. Where maximal models fail to converge, we use bounded
linear mixed models (\texttt{blmer}/\texttt{bglmer}) as a first remedy.
If singularity persists, we iteratively simplify the random effects
structure, removing random slopes one at a time and retaining the most
complex structure that converges without singularity. Among these
converging models, we do further AIC-based selection to find the most
simple effects structure without increasing AIC

\textbf{Fixed effects selection.} Starting with the selected random
effects structure and a full fixed effects model including surprisal,
previous positional entropy, their interaction, and repetition, we
similarly use AIC-based stepwise backward selection to prune the fixed
effects. We first test whether repetition can be removed without
increasing AIC. We then test the interaction term between surprisal and
previous position entropy, followed by individual main effects if the
interaction is not retained. An effect is retained if removing it
increases AIC.

\section{Results}\label{results}

\begin{Shaded}
\begin{Highlighting}[]
\NormalTok{df.m\_h0 }\OtherTok{\textless{}{-}}\NormalTok{ df}
\NormalTok{df.m\_h0.correct }\OtherTok{\textless{}{-}}\NormalTok{ df.correct}
\end{Highlighting}
\end{Shaded}

\begin{Shaded}
\begin{Highlighting}[]
\NormalTok{m\_h0\_rt }\OtherTok{\textless{}{-}}
    \FunctionTok{lmer}\NormalTok{(log\_rt }\SpecialCharTok{\textasciitilde{}}\NormalTok{ block\_c }\SpecialCharTok{*}\NormalTok{ is\_repetition }\SpecialCharTok{+}\NormalTok{ (block\_c }\SpecialCharTok{*}\NormalTok{ is\_repetition }\SpecialCharTok{|}\NormalTok{ subject\_id) }\SpecialCharTok{+}\NormalTok{ (}\DecValTok{1} \SpecialCharTok{|}\NormalTok{ position),}
        \AttributeTok{data =}\NormalTok{ df.m\_h0.correct,}
        \AttributeTok{REML =} \ConstantTok{FALSE}\NormalTok{,}
        \AttributeTok{control =} \FunctionTok{lmerControl}\NormalTok{(}\AttributeTok{optimizer =} \StringTok{"bobyqa"}\NormalTok{)}
\NormalTok{    ) }\CommentTok{\# can change}

\FunctionTok{summary}\NormalTok{(m\_h0\_rt)}

\NormalTok{df.m\_h0.correct }\OtherTok{\textless{}{-}}\NormalTok{ df.correct }\SpecialCharTok{\%\textgreater{}\%}
    \FunctionTok{group\_by}\NormalTok{(matrix\_size)}

\NormalTok{m\_h0\_acc }\OtherTok{\textless{}{-}}
    \FunctionTok{glmer}\NormalTok{(correct }\SpecialCharTok{\textasciitilde{}}\NormalTok{ block\_c\_ }\SpecialCharTok{*}\NormalTok{ is\_repetition }\SpecialCharTok{+}\NormalTok{ (block\_c\_ }\SpecialCharTok{*}\NormalTok{ is\_repetition }\SpecialCharTok{|}\NormalTok{ subject\_id) }\SpecialCharTok{+}\NormalTok{ (}\DecValTok{1} \SpecialCharTok{|}\NormalTok{ position),}
        \AttributeTok{data =}\NormalTok{ df.m\_h0,}
        \AttributeTok{family =}\NormalTok{ binomial,}
        \AttributeTok{control =} \FunctionTok{glmerControl}\NormalTok{(}\AttributeTok{optimizer =} \StringTok{"bobyqa"}\NormalTok{)}
\NormalTok{    ) }\CommentTok{\# can change}

\FunctionTok{summary}\NormalTok{(m\_h0\_acc)}
\end{Highlighting}
\end{Shaded}

As a preliminary sanity check, we confirmed that RTs decreased across
blocks (\(\beta\) = -X.XX, p \textless{} .001), establishing that
performance improvement occurred.

\subsection{Surprisal and Previous Position Entropy
Effects}\label{surprisal-and-previous-position-entropy-effects}

\begin{Shaded}
\begin{Highlighting}[]
\CommentTok{\# (filter for last block only, for each matrix\_size)}

\NormalTok{df.m\_h1h2.finalblock }\OtherTok{\textless{}{-}}\NormalTok{ df }\SpecialCharTok{\%\textgreater{}\%}
    \FunctionTok{filter}\NormalTok{(block }\SpecialCharTok{==}\NormalTok{ EXPERIMENT\_CONFIG[[}\StringTok{"n\_blocks"}\NormalTok{]] }\SpecialCharTok{{-}} \DecValTok{1}\NormalTok{)}

\NormalTok{data\_list }\OtherTok{\textless{}{-}} \FunctionTok{split}\NormalTok{(df.m\_h1h2.finalblock, df.m\_h1h2.finalblock}\SpecialCharTok{$}\NormalTok{matrix\_size)}

\NormalTok{models }\OtherTok{\textless{}{-}} \FunctionTok{map}\NormalTok{(data\_list, }\SpecialCharTok{\textasciitilde{}} \FunctionTok{lmer}\NormalTok{(}
\NormalTok{    log\_rt }\SpecialCharTok{\textasciitilde{}}\NormalTok{ surprisal\_c }\SpecialCharTok{*}\NormalTok{ prev\_entropy\_c }\SpecialCharTok{+}\NormalTok{ is\_repetition }\SpecialCharTok{+}
\NormalTok{        (surprisal\_c }\SpecialCharTok{*}\NormalTok{ prev\_entropy\_c }\SpecialCharTok{||}\NormalTok{ subject\_id) }\SpecialCharTok{+}\NormalTok{ (}\DecValTok{1} \SpecialCharTok{|}\NormalTok{ position),}
    \AttributeTok{data =}\NormalTok{ .x,}
    \AttributeTok{REML =} \ConstantTok{FALSE}\NormalTok{,}
    \AttributeTok{control =} \FunctionTok{lmerControl}\NormalTok{(}\AttributeTok{optimizer =} \StringTok{"bobyqa"}\NormalTok{)}
\NormalTok{))}

\NormalTok{model\_summaries }\OtherTok{\textless{}{-}} \FunctionTok{map}\NormalTok{(models, summary)}
\NormalTok{model\_summaries[}\DecValTok{2}\NormalTok{]}
\end{Highlighting}
\end{Shaded}

\begin{Shaded}
\begin{Highlighting}[]
\NormalTok{model\_structures\_random\_slope }\OtherTok{\textless{}{-}} \FunctionTok{list}\NormalTok{(}
    \AttributeTok{base =} \StringTok{"(surprisal\_c * prev\_entropy\_c || subject\_id) + (1 | position)"}\NormalTok{,}
    \AttributeTok{main =} \StringTok{"(surprisal\_c + prev\_entropy\_c || subject\_id) + (1 | position)"}\NormalTok{,}
    \AttributeTok{s\_only =} \StringTok{"(surprisal\_c | subject\_id) + (1 | position)"}\NormalTok{,}
    \AttributeTok{e\_only =} \StringTok{"(prev\_entropy\_c | subject\_id) + (1 | position)"}\NormalTok{,}
    \AttributeTok{intercept =} \StringTok{"(1 | subject\_id) + (1 | position)"}
\NormalTok{)}

\NormalTok{all\_models\_random\_slope }\OtherTok{\textless{}{-}} \FunctionTok{map}\NormalTok{(}
\NormalTok{    model\_structures\_random\_slope,}
    \SpecialCharTok{\textasciitilde{}} \FunctionTok{map}\NormalTok{(data\_list, }\ControlFlowTok{function}\NormalTok{(dat) \{}
        \FunctionTok{lmer}\NormalTok{(}
            \FunctionTok{as.formula}\NormalTok{(}
                \FunctionTok{paste0}\NormalTok{(}\StringTok{"log\_rt \textasciitilde{} surprisal\_c * prev\_entropy\_c + is\_repetition + "}\NormalTok{, .x)}
\NormalTok{            ),}
            \AttributeTok{data =}\NormalTok{ dat,}
            \AttributeTok{REML =} \ConstantTok{FALSE}\NormalTok{,}
            \AttributeTok{control =} \FunctionTok{lmerControl}\NormalTok{(}\AttributeTok{optimizer =} \StringTok{"bobyqa"}\NormalTok{)}
\NormalTok{        )}
\NormalTok{    \})}
\NormalTok{)}

\NormalTok{aic\_table\_random\_slope }\OtherTok{\textless{}{-}} \FunctionTok{map\_dfr}\NormalTok{(}
    \FunctionTok{names}\NormalTok{(all\_models\_random\_slope),}
    \ControlFlowTok{function}\NormalTok{(model\_name) \{}
        \FunctionTok{map\_dfr}\NormalTok{(}
            \FunctionTok{names}\NormalTok{(all\_models\_random\_slope[[model\_name]]),}
            \ControlFlowTok{function}\NormalTok{(cond) \{}
\NormalTok{                m }\OtherTok{\textless{}{-}}\NormalTok{ all\_models\_random\_slope[[model\_name]][[cond]]}
                \FunctionTok{data.frame}\NormalTok{(}
                    \AttributeTok{model =}\NormalTok{ model\_name,}
                    \AttributeTok{matrix\_size =}\NormalTok{ cond,}
                    \AttributeTok{AIC =} \FunctionTok{AIC}\NormalTok{(m),}
                    \AttributeTok{isSingular =} \FunctionTok{isSingular}\NormalTok{(m)}
\NormalTok{                )}
\NormalTok{            \}}
\NormalTok{        )}
\NormalTok{    \}}
\NormalTok{)}

\NormalTok{aic\_table\_random\_slope }\SpecialCharTok{\%\textgreater{}\%}
    \FunctionTok{filter}\NormalTok{(aic\_table\_random\_slope}\SpecialCharTok{$}\NormalTok{isSingular }\SpecialCharTok{==} \ConstantTok{FALSE}\NormalTok{) }\SpecialCharTok{\%\textgreater{}\%}
    \FunctionTok{group\_by}\NormalTok{(model) }\SpecialCharTok{\%\textgreater{}\%}
    \FunctionTok{summarise}\NormalTok{(}\AttributeTok{total\_AIC =} \FunctionTok{sum}\NormalTok{(AIC)) }\SpecialCharTok{\%\textgreater{}\%}
    \FunctionTok{arrange}\NormalTok{(total\_AIC)}
\CommentTok{\# group\_by(matrix\_size) \%\textgreater{}\%}
\CommentTok{\# arrange(matrix\_size, AIC)}
\end{Highlighting}
\end{Shaded}

\begin{Shaded}
\begin{Highlighting}[]
\NormalTok{model\_structures\_fixed\_effect }\OtherTok{\textless{}{-}} \FunctionTok{list}\NormalTok{(}
    \AttributeTok{base =} \StringTok{"surprisal\_c * prev\_entropy\_c + is\_repetition"}\NormalTok{,}
    \AttributeTok{no\_interaction =} \StringTok{"surprisal\_c + prev\_entropy\_c + is\_repetition"}\NormalTok{,}
    \AttributeTok{no\_is\_repetition =} \StringTok{"surprisal\_c * prev\_entropy\_c"}\NormalTok{,}
    \AttributeTok{only\_main =} \StringTok{"surprisal\_c + prev\_entropy\_c"}\NormalTok{,}
    \AttributeTok{surprisal\_is\_repetition =} \StringTok{"surprisal\_c + is\_repetition"}\NormalTok{,}
    \AttributeTok{entropy\_is\_repetition =} \StringTok{"prev\_entropy\_c + is\_repetition"}
\NormalTok{)}

\NormalTok{all\_models\_fixed\_effect }\OtherTok{\textless{}{-}} \FunctionTok{map}\NormalTok{(}
\NormalTok{    model\_structures\_fixed\_effect,}
    \SpecialCharTok{\textasciitilde{}} \FunctionTok{map}\NormalTok{(data\_list, }\ControlFlowTok{function}\NormalTok{(dat) \{}
        \FunctionTok{lmer}\NormalTok{(}
            \FunctionTok{as.formula}\NormalTok{(}
                \FunctionTok{paste0}\NormalTok{(}\StringTok{"log\_rt \textasciitilde{} "}\NormalTok{, .x, }\StringTok{" + (surprisal\_c * prev\_entropy\_c || subject\_id) + (1 | position)"}\NormalTok{)}
\NormalTok{            ),}
            \AttributeTok{data =}\NormalTok{ dat,}
            \AttributeTok{REML =} \ConstantTok{FALSE}\NormalTok{,}
            \AttributeTok{control =} \FunctionTok{lmerControl}\NormalTok{(}\AttributeTok{optimizer =} \StringTok{"bobyqa"}\NormalTok{)}
\NormalTok{        )}
\NormalTok{    \})}
\NormalTok{)}

\NormalTok{aic\_table\_fixed\_effect }\OtherTok{\textless{}{-}} \FunctionTok{map\_dfr}\NormalTok{(}
    \FunctionTok{names}\NormalTok{(all\_models\_fixed\_effect),}
    \ControlFlowTok{function}\NormalTok{(model\_name) \{}
        \FunctionTok{map\_dfr}\NormalTok{(}
            \FunctionTok{names}\NormalTok{(all\_models\_fixed\_effect[[model\_name]]),}
            \ControlFlowTok{function}\NormalTok{(cond) \{}
\NormalTok{                m }\OtherTok{\textless{}{-}}\NormalTok{ all\_models\_fixed\_effect[[model\_name]][[cond]]}
                \FunctionTok{data.frame}\NormalTok{(}
                    \AttributeTok{model =}\NormalTok{ model\_name,}
                    \AttributeTok{matrix\_size =}\NormalTok{ cond,}
                    \AttributeTok{AIC =} \FunctionTok{AIC}\NormalTok{(m),}
                    \AttributeTok{isSingular =} \FunctionTok{isSingular}\NormalTok{(m)}
\NormalTok{                )}
\NormalTok{            \}}
\NormalTok{        )}
\NormalTok{    \}}
\NormalTok{)}

\NormalTok{aic\_table\_fixed\_effect }\SpecialCharTok{\%\textgreater{}\%}
    \FunctionTok{filter}\NormalTok{(aic\_table\_fixed\_effect}\SpecialCharTok{$}\NormalTok{isSingular }\SpecialCharTok{==} \ConstantTok{FALSE}\NormalTok{) }\SpecialCharTok{\%\textgreater{}\%}
    \FunctionTok{group\_by}\NormalTok{(model) }\SpecialCharTok{\%\textgreater{}\%}
    \FunctionTok{summarise}\NormalTok{(}\AttributeTok{total\_AIC =} \FunctionTok{sum}\NormalTok{(AIC)) }\SpecialCharTok{\%\textgreater{}\%}
    \FunctionTok{arrange}\NormalTok{(total\_AIC) }
\end{Highlighting}
\end{Shaded}

\begin{Shaded}
\begin{Highlighting}[]
\FunctionTok{theme\_set}\NormalTok{(}\FunctionTok{theme\_minimal}\NormalTok{())}

\NormalTok{effects\_df }\OtherTok{\textless{}{-}} \FunctionTok{map\_dfr}\NormalTok{(}
    \FunctionTok{names}\NormalTok{(models),}
    \SpecialCharTok{\textasciitilde{}}\NormalTok{ broom.mixed}\SpecialCharTok{::}\FunctionTok{tidy}\NormalTok{(models[[.x]]) }\SpecialCharTok{\%\textgreater{}\%}
        \FunctionTok{filter}\NormalTok{(effect }\SpecialCharTok{==} \StringTok{"fixed"} \SpecialCharTok{\&}\NormalTok{ term }\SpecialCharTok{!=} \StringTok{"(Intercept)"}\NormalTok{) }\SpecialCharTok{\%\textgreater{}\%}
        \FunctionTok{mutate}\NormalTok{(}\AttributeTok{matrix\_size\_c =}\NormalTok{ .x)}
\NormalTok{) }\SpecialCharTok{\%\textgreater{}\%}
    \FunctionTok{mutate}\NormalTok{(}
        \AttributeTok{sig\_label =} \FunctionTok{case\_when}\NormalTok{(}
\NormalTok{            p.value }\SpecialCharTok{\textless{}} \FloatTok{0.001} \SpecialCharTok{\textasciitilde{}} \StringTok{"***"}\NormalTok{,}
\NormalTok{            p.value }\SpecialCharTok{\textless{}} \FloatTok{0.01} \SpecialCharTok{\textasciitilde{}} \StringTok{"**"}\NormalTok{,}
\NormalTok{            p.value }\SpecialCharTok{\textless{}} \FloatTok{0.05} \SpecialCharTok{\textasciitilde{}} \StringTok{"*"}\NormalTok{,}
            \ConstantTok{TRUE} \SpecialCharTok{\textasciitilde{}} \StringTok{""}
\NormalTok{        )}
\NormalTok{    )}

\CommentTok{\# effects\_df$matrix\_size \textless{}{-} as.numeric(effects\_df$matrix\_size)}

\NormalTok{p\_h1h2\_fixed\_effects\_finalblock }\OtherTok{\textless{}{-}} \FunctionTok{ggplot}\NormalTok{(effects\_df, }\FunctionTok{aes}\NormalTok{(}\AttributeTok{x =}\NormalTok{ matrix\_size\_c, }\AttributeTok{y =}\NormalTok{ estimate, }\AttributeTok{color =}\NormalTok{ term, }\AttributeTok{group =}\NormalTok{ term)) }\SpecialCharTok{+}
    \FunctionTok{geom\_line}\NormalTok{() }\SpecialCharTok{+}
    \FunctionTok{geom\_point}\NormalTok{() }\SpecialCharTok{+}
    \FunctionTok{geom\_errorbar}\NormalTok{(}\FunctionTok{aes}\NormalTok{(}\AttributeTok{ymin =}\NormalTok{ estimate }\SpecialCharTok{{-}}\NormalTok{ std.error, }\AttributeTok{ymax =}\NormalTok{ estimate }\SpecialCharTok{+}\NormalTok{ std.error), }\AttributeTok{width =} \FloatTok{0.2}\NormalTok{) }\SpecialCharTok{+}
    \FunctionTok{geom\_text}\NormalTok{(}\FunctionTok{aes}\NormalTok{(}\AttributeTok{label =}\NormalTok{ sig\_label), }\AttributeTok{vjust =} \SpecialCharTok{{-}}\FloatTok{1.2}\NormalTok{, }\AttributeTok{size =} \DecValTok{5}\NormalTok{, }\AttributeTok{show.legend =} \ConstantTok{FALSE}\NormalTok{)}
\FunctionTok{labs}\NormalTok{(}\AttributeTok{x =} \StringTok{"Matrix Size (zero{-}centered)"}\NormalTok{, }\AttributeTok{y =} \StringTok{"Effect Estimate"}\NormalTok{, }\AttributeTok{color =} \StringTok{"Parameter"}\NormalTok{) }\SpecialCharTok{+}
    \FunctionTok{theme\_minimal}\NormalTok{()}
\FunctionTok{ggsave}\NormalTok{(p\_h1h2\_fixed\_effects\_finalblock, }\AttributeTok{file =} \StringTok{"../plots/h1h2\_finalblock.png"}\NormalTok{)}
\end{Highlighting}
\end{Shaded}

\begin{Shaded}
\begin{Highlighting}[]
\NormalTok{cur\_condition }\OtherTok{\textless{}{-}} \DecValTok{8}
\NormalTok{data\_m4 }\OtherTok{\textless{}{-}} \FunctionTok{subset}\NormalTok{(df, matrix\_size }\SpecialCharTok{==}\NormalTok{ cur\_condition) }\CommentTok{\# can change}

\NormalTok{data\_list\_block }\OtherTok{\textless{}{-}} \FunctionTok{split}\NormalTok{(data\_m4, data\_m4}\SpecialCharTok{$}\NormalTok{block)}

\CommentTok{\# random effects: (prev\_entropy\_c | subject\_id) + (1 | position) no singular fit for all conditions}
\NormalTok{models\_block }\OtherTok{\textless{}{-}} \FunctionTok{map2}\NormalTok{(}
    \FunctionTok{names}\NormalTok{(data\_list\_block),}
\NormalTok{    data\_list\_block,}
    \ControlFlowTok{function}\NormalTok{(name, dat) \{}
\NormalTok{        model\_path }\OtherTok{\textless{}{-}} \FunctionTok{paste0}\NormalTok{(}\StringTok{"./models/per\_block\_condition\_"}\NormalTok{, cur\_condition, }\StringTok{"/model\_block\_"}\NormalTok{, name, }\StringTok{".rds"}\NormalTok{)}
\NormalTok{        dir\_path }\OtherTok{\textless{}{-}} \FunctionTok{dirname}\NormalTok{(model\_path)}
        \ControlFlowTok{if}\NormalTok{ (}\SpecialCharTok{!}\FunctionTok{dir.exists}\NormalTok{(dir\_path)) }\FunctionTok{dir.create}\NormalTok{(dir\_path, }\AttributeTok{recursive =} \ConstantTok{TRUE}\NormalTok{)}
        \ControlFlowTok{if}\NormalTok{ (}\FunctionTok{file.exists}\NormalTok{(model\_path)) \{}
            \FunctionTok{readRDS}\NormalTok{(model\_path)}
\NormalTok{        \} }\ControlFlowTok{else}\NormalTok{ \{}
\NormalTok{            m }\OtherTok{\textless{}{-}} \FunctionTok{lmer}\NormalTok{(}
\NormalTok{                log\_rt }\SpecialCharTok{\textasciitilde{}}\NormalTok{ surprisal\_c }\SpecialCharTok{*}\NormalTok{ prev\_entropy\_c }\SpecialCharTok{+}\NormalTok{ is\_repetition }\SpecialCharTok{+}
\NormalTok{                    (prev\_entropy\_c }\SpecialCharTok{|}\NormalTok{ subject\_id) }\SpecialCharTok{+}\NormalTok{ (}\DecValTok{1} \SpecialCharTok{|}\NormalTok{ position),}
                \AttributeTok{data =}\NormalTok{ dat,}
                \AttributeTok{REML =} \ConstantTok{FALSE}\NormalTok{,}
                \AttributeTok{control =} \FunctionTok{lmerControl}\NormalTok{(}\AttributeTok{optimizer =} \StringTok{"bobyqa"}\NormalTok{)}
\NormalTok{            )}
            \FunctionTok{saveRDS}\NormalTok{(m, model\_path)}
\NormalTok{            m}
\NormalTok{        \}}
\NormalTok{    \}}
\NormalTok{)}

\NormalTok{model\_summaries\_block }\OtherTok{\textless{}{-}} \FunctionTok{map}\NormalTok{(models\_block, summary)}
\NormalTok{model\_summaries\_block[}\DecValTok{10}\NormalTok{]}
\end{Highlighting}
\end{Shaded}

\begin{verbatim}
[[1]]
Linear mixed model fit by maximum likelihood . t-tests use Satterthwaite's
  method [lmerModLmerTest]
Formula: 
log_rt ~ surprisal_c * prev_entropy_c + is_repetition + (prev_entropy_c |  
    subject_id) + (1 | position)
   Data: dat
Control: lmerControl(optimizer = "bobyqa")

     AIC      BIC   logLik deviance df.resid 
   466.3    524.3   -223.1    446.3     2454 

Scaled residuals: 
    Min      1Q  Median      3Q     Max 
-5.2700 -0.5975 -0.1025  0.4629  7.0421 

Random effects:
 Groups     Name           Variance Std.Dev. Corr 
 subject_id (Intercept)    0.029647 0.17218       
            prev_entropy_c 0.004654 0.06822  -0.34
 position   (Intercept)    0.003398 0.05829       
 Residual                  0.065300 0.25554       
Number of obs: 2464, groups:  subject_id, 32; position, 8

Fixed effects:
                             Estimate Std. Error         df t value Pr(>|t|)
(Intercept)                 6.420e+00  3.739e-02  3.466e+01 171.696  < 2e-16
surprisal_c                 1.661e-02  6.202e-03  2.415e+03   2.677  0.00748
prev_entropy_c              2.431e-02  1.530e-02  4.677e+01   1.588  0.11899
is_repetitionTRUE          -3.072e-01  1.629e-02  2.413e+03 -18.860  < 2e-16
surprisal_c:prev_entropy_c  1.494e-03  6.940e-03  2.426e+03   0.215  0.82955
                              
(Intercept)                ***
surprisal_c                ** 
prev_entropy_c                
is_repetitionTRUE          ***
surprisal_c:prev_entropy_c    
---
Signif. codes:  0 '***' 0.001 '**' 0.01 '*' 0.05 '.' 0.1 ' ' 1

Correlation of Fixed Effects:
            (Intr) srprs_ prv_n_ i_TRUE
surprisal_c -0.026                     
prv_ntrpy_c -0.227 -0.342              
is_rpttTRUE -0.058  0.011 -0.109       
srprsl_c:__ -0.107  0.232  0.138  0.033
\end{verbatim}

\begin{Shaded}
\begin{Highlighting}[]
\NormalTok{model\_structures\_random\_slope }\OtherTok{\textless{}{-}} \FunctionTok{list}\NormalTok{(}
    \AttributeTok{base =} \StringTok{"(prev\_entropy\_c | subject\_id) + (1 | position)"}\NormalTok{,}
    \AttributeTok{main =} \StringTok{"(surprisal\_c + prev\_entropy\_c || subject\_id) + (1 | position)"}\NormalTok{,}
    \AttributeTok{s\_only =} \StringTok{"(surprisal\_c | subject\_id) + (1 | position)"}\NormalTok{,}
    \AttributeTok{e\_only =} \StringTok{"(prev\_entropy\_c | subject\_id) + (1 | position)"}\NormalTok{,}
    \AttributeTok{intercept =} \StringTok{"(1 | subject\_id) + (1 | position)"}
\NormalTok{)}

\NormalTok{all\_models\_random\_slope }\OtherTok{\textless{}{-}} \FunctionTok{map}\NormalTok{(}
\NormalTok{    model\_structures\_random\_slope,}
    \SpecialCharTok{\textasciitilde{}} \FunctionTok{map}\NormalTok{(data\_list, }\ControlFlowTok{function}\NormalTok{(dat) \{}
        \FunctionTok{lmer}\NormalTok{(}
            \FunctionTok{as.formula}\NormalTok{(}
                \FunctionTok{paste0}\NormalTok{(}\StringTok{"log\_rt \textasciitilde{} surprisal\_c * prev\_entropy\_c + is\_repetition + "}\NormalTok{, .x)}
\NormalTok{            ),}
            \AttributeTok{data =}\NormalTok{ dat,}
            \AttributeTok{REML =} \ConstantTok{FALSE}\NormalTok{,}
            \AttributeTok{control =} \FunctionTok{lmerControl}\NormalTok{(}\AttributeTok{optimizer =} \StringTok{"bobyqa"}\NormalTok{)}
\NormalTok{        )}
\NormalTok{    \})}
\NormalTok{)}

\NormalTok{aic\_table\_random\_slope }\OtherTok{\textless{}{-}} \FunctionTok{map\_dfr}\NormalTok{(}
    \FunctionTok{names}\NormalTok{(all\_models\_random\_slope),}
    \ControlFlowTok{function}\NormalTok{(model\_name) \{}
        \FunctionTok{map\_dfr}\NormalTok{(}
            \FunctionTok{names}\NormalTok{(all\_models\_random\_slope[[model\_name]]),}
            \ControlFlowTok{function}\NormalTok{(cond) \{}
\NormalTok{                m }\OtherTok{\textless{}{-}}\NormalTok{ all\_models\_random\_slope[[model\_name]][[cond]]}
                \FunctionTok{data.frame}\NormalTok{(}
                    \AttributeTok{model =}\NormalTok{ model\_name,}
                    \AttributeTok{matrix\_size =}\NormalTok{ cond,}
                    \AttributeTok{AIC =} \FunctionTok{AIC}\NormalTok{(m),}
                    \AttributeTok{isSingular =} \FunctionTok{isSingular}\NormalTok{(m)}
\NormalTok{                )}
\NormalTok{            \}}
\NormalTok{        )}
\NormalTok{    \}}
\NormalTok{)}

\NormalTok{aic\_table\_random\_slope }\SpecialCharTok{\%\textgreater{}\%}
    \FunctionTok{filter}\NormalTok{(aic\_table\_random\_slope}\SpecialCharTok{$}\NormalTok{isSingular }\SpecialCharTok{==} \ConstantTok{FALSE}\NormalTok{) }\SpecialCharTok{\%\textgreater{}\%}
    \FunctionTok{group\_by}\NormalTok{(model) }\SpecialCharTok{\%\textgreater{}\%}
    \FunctionTok{summarise}\NormalTok{(}\AttributeTok{total\_AIC =} \FunctionTok{sum}\NormalTok{(AIC)) }\SpecialCharTok{\%\textgreater{}\%}
    \FunctionTok{arrange}\NormalTok{(total\_AIC)}
\end{Highlighting}
\end{Shaded}

\begin{verbatim}
# A tibble: 5 x 2
  model     total_AIC
  <chr>         <dbl>
1 main          1536.
2 base          1564.
3 e_only        1564.
4 s_only        1653.
5 intercept     2016.
\end{verbatim}

\begin{Shaded}
\begin{Highlighting}[]
\CommentTok{\# group\_by(matrix\_size) \%\textgreater{}\%}
\CommentTok{\# arrange(matrix\_size, AIC)}
\end{Highlighting}
\end{Shaded}

\begin{Shaded}
\begin{Highlighting}[]
\FunctionTok{theme}\NormalTok{(}\AttributeTok{set=}\FunctionTok{theme\_minimal}\NormalTok{())}
\end{Highlighting}
\end{Shaded}

\begin{verbatim}
<theme> List of 1
 $ set: <theme> List of 144
 .. $ line                            : <ggplot2::element_line>
 ..  ..@ colour       : chr "black"
 ..  ..@ linewidth    : num 0.5
 ..  ..@ linetype     : num 1
 ..  ..@ lineend      : chr "butt"
 ..  ..@ linejoin     : chr "round"
 ..  ..@ arrow        : logi FALSE
 ..  ..@ arrow.fill   : chr "black"
 ..  ..@ inherit.blank: logi TRUE
 .. $ rect                            : <ggplot2::element_rect>
 ..  ..@ fill         : chr "white"
 ..  ..@ colour       : chr "black"
 ..  ..@ linewidth    : num 0.5
 ..  ..@ linetype     : num 1
 ..  ..@ linejoin     : chr "round"
 ..  ..@ inherit.blank: logi TRUE
 .. $ text                            : <ggplot2::element_text>
 ..  ..@ family       : chr ""
 ..  ..@ face         : chr "plain"
 ..  ..@ italic       : chr NA
 ..  ..@ fontweight   : num NA
 ..  ..@ fontwidth    : num NA
 ..  ..@ colour       : chr "black"
 ..  ..@ size         : num 11
 ..  ..@ hjust        : num 0.5
 ..  ..@ vjust        : num 0.5
 ..  ..@ angle        : num 0
 ..  ..@ lineheight   : num 0.9
 ..  ..@ margin       : <ggplot2::margin> num [1:4] 0 0 0 0
 ..  ..@ debug        : logi FALSE
 ..  ..@ inherit.blank: logi TRUE
 .. $ title                           : <ggplot2::element_text>
 ..  ..@ family       : NULL
 ..  ..@ face         : NULL
 ..  ..@ italic       : chr NA
 ..  ..@ fontweight   : num NA
 ..  ..@ fontwidth    : num NA
 ..  ..@ colour       : NULL
 ..  ..@ size         : NULL
 ..  ..@ hjust        : NULL
 ..  ..@ vjust        : NULL
 ..  ..@ angle        : NULL
 ..  ..@ lineheight   : NULL
 ..  ..@ margin       : NULL
 ..  ..@ debug        : NULL
 ..  ..@ inherit.blank: logi TRUE
 .. $ point                           : <ggplot2::element_point>
 ..  ..@ colour       : chr "black"
 ..  ..@ shape        : num 19
 ..  ..@ size         : num 1.5
 ..  ..@ fill         : chr "white"
 ..  ..@ stroke       : num 0.5
 ..  ..@ inherit.blank: logi TRUE
 .. $ polygon                         : <ggplot2::element_polygon>
 ..  ..@ fill         : chr "white"
 ..  ..@ colour       : chr "black"
 ..  ..@ linewidth    : num 0.5
 ..  ..@ linetype     : num 1
 ..  ..@ linejoin     : chr "round"
 ..  ..@ inherit.blank: logi TRUE
 .. $ geom                            : <ggplot2::element_geom>
 ..  ..@ ink        : chr "black"
 ..  ..@ paper      : chr "white"
 ..  ..@ accent     : chr "#3366FF"
 ..  ..@ linewidth  : num 0.5
 ..  ..@ borderwidth: num 0.5
 ..  ..@ linetype   : int 1
 ..  ..@ bordertype : int 1
 ..  ..@ family     : chr ""
 ..  ..@ fontsize   : num 3.87
 ..  ..@ pointsize  : num 1.5
 ..  ..@ pointshape : num 19
 ..  ..@ colour     : NULL
 ..  ..@ fill       : NULL
 .. $ spacing                         : 'simpleUnit' num 5.5points
 ..  ..- attr(*, "unit")= int 8
 .. $ margins                         : <ggplot2::margin> num [1:4] 5.5 5.5 5.5 5.5
 .. $ aspect.ratio                    : NULL
 .. $ axis.title                      : NULL
 .. $ axis.title.x                    : <ggplot2::element_text>
 ..  ..@ family       : NULL
 ..  ..@ face         : NULL
 ..  ..@ italic       : chr NA
 ..  ..@ fontweight   : num NA
 ..  ..@ fontwidth    : num NA
 ..  ..@ colour       : NULL
 ..  ..@ size         : NULL
 ..  ..@ hjust        : NULL
 ..  ..@ vjust        : num 1
 ..  ..@ angle        : NULL
 ..  ..@ lineheight   : NULL
 ..  ..@ margin       : <ggplot2::margin> num [1:4] 2.75 0 0 0
 ..  ..@ debug        : NULL
 ..  ..@ inherit.blank: logi TRUE
 .. $ axis.title.x.top                : <ggplot2::element_text>
 ..  ..@ family       : NULL
 ..  ..@ face         : NULL
 ..  ..@ italic       : chr NA
 ..  ..@ fontweight   : num NA
 ..  ..@ fontwidth    : num NA
 ..  ..@ colour       : NULL
 ..  ..@ size         : NULL
 ..  ..@ hjust        : NULL
 ..  ..@ vjust        : num 0
 ..  ..@ angle        : NULL
 ..  ..@ lineheight   : NULL
 ..  ..@ margin       : <ggplot2::margin> num [1:4] 0 0 2.75 0
 ..  ..@ debug        : NULL
 ..  ..@ inherit.blank: logi TRUE
 .. $ axis.title.x.bottom             : NULL
 .. $ axis.title.y                    : <ggplot2::element_text>
 ..  ..@ family       : NULL
 ..  ..@ face         : NULL
 ..  ..@ italic       : chr NA
 ..  ..@ fontweight   : num NA
 ..  ..@ fontwidth    : num NA
 ..  ..@ colour       : NULL
 ..  ..@ size         : NULL
 ..  ..@ hjust        : NULL
 ..  ..@ vjust        : num 1
 ..  ..@ angle        : num 90
 ..  ..@ lineheight   : NULL
 ..  ..@ margin       : <ggplot2::margin> num [1:4] 0 2.75 0 0
 ..  ..@ debug        : NULL
 ..  ..@ inherit.blank: logi TRUE
 .. $ axis.title.y.left               : NULL
 .. $ axis.title.y.right              : <ggplot2::element_text>
 ..  ..@ family       : NULL
 ..  ..@ face         : NULL
 ..  ..@ italic       : chr NA
 ..  ..@ fontweight   : num NA
 ..  ..@ fontwidth    : num NA
 ..  ..@ colour       : NULL
 ..  ..@ size         : NULL
 ..  ..@ hjust        : NULL
 ..  ..@ vjust        : num 1
 ..  ..@ angle        : num -90
 ..  ..@ lineheight   : NULL
 ..  ..@ margin       : <ggplot2::margin> num [1:4] 0 0 0 2.75
 ..  ..@ debug        : NULL
 ..  ..@ inherit.blank: logi TRUE
 .. $ axis.text                       : <ggplot2::element_text>
 ..  ..@ family       : NULL
 ..  ..@ face         : NULL
 ..  ..@ italic       : chr NA
 ..  ..@ fontweight   : num NA
 ..  ..@ fontwidth    : num NA
 ..  ..@ colour       : chr "#4D4D4DFF"
 ..  ..@ size         : 'rel' num 0.8
 ..  ..@ hjust        : NULL
 ..  ..@ vjust        : NULL
 ..  ..@ angle        : NULL
 ..  ..@ lineheight   : NULL
 ..  ..@ margin       : NULL
 ..  ..@ debug        : NULL
 ..  ..@ inherit.blank: logi TRUE
 .. $ axis.text.x                     : <ggplot2::element_text>
 ..  ..@ family       : NULL
 ..  ..@ face         : NULL
 ..  ..@ italic       : chr NA
 ..  ..@ fontweight   : num NA
 ..  ..@ fontwidth    : num NA
 ..  ..@ colour       : NULL
 ..  ..@ size         : NULL
 ..  ..@ hjust        : NULL
 ..  ..@ vjust        : num 1
 ..  ..@ angle        : NULL
 ..  ..@ lineheight   : NULL
 ..  ..@ margin       : <ggplot2::margin> num [1:4] 2.2 0 0 0
 ..  ..@ debug        : NULL
 ..  ..@ inherit.blank: logi TRUE
 .. $ axis.text.x.top                 : <ggplot2::element_text>
 ..  ..@ family       : NULL
 ..  ..@ face         : NULL
 ..  ..@ italic       : chr NA
 ..  ..@ fontweight   : num NA
 ..  ..@ fontwidth    : num NA
 ..  ..@ colour       : NULL
 ..  ..@ size         : NULL
 ..  ..@ hjust        : NULL
 ..  ..@ vjust        : NULL
 ..  ..@ angle        : NULL
 ..  ..@ lineheight   : NULL
 ..  ..@ margin       : <ggplot2::margin> num [1:4] 0 0 4.95 0
 ..  ..@ debug        : NULL
 ..  ..@ inherit.blank: logi TRUE
 .. $ axis.text.x.bottom              : <ggplot2::element_text>
 ..  ..@ family       : NULL
 ..  ..@ face         : NULL
 ..  ..@ italic       : chr NA
 ..  ..@ fontweight   : num NA
 ..  ..@ fontwidth    : num NA
 ..  ..@ colour       : NULL
 ..  ..@ size         : NULL
 ..  ..@ hjust        : NULL
 ..  ..@ vjust        : NULL
 ..  ..@ angle        : NULL
 ..  ..@ lineheight   : NULL
 ..  ..@ margin       : <ggplot2::margin> num [1:4] 4.95 0 0 0
 ..  ..@ debug        : NULL
 ..  ..@ inherit.blank: logi TRUE
 .. $ axis.text.y                     : <ggplot2::element_text>
 ..  ..@ family       : NULL
 ..  ..@ face         : NULL
 ..  ..@ italic       : chr NA
 ..  ..@ fontweight   : num NA
 ..  ..@ fontwidth    : num NA
 ..  ..@ colour       : NULL
 ..  ..@ size         : NULL
 ..  ..@ hjust        : num 1
 ..  ..@ vjust        : NULL
 ..  ..@ angle        : NULL
 ..  ..@ lineheight   : NULL
 ..  ..@ margin       : <ggplot2::margin> num [1:4] 0 2.2 0 0
 ..  ..@ debug        : NULL
 ..  ..@ inherit.blank: logi TRUE
 .. $ axis.text.y.left                : <ggplot2::element_text>
 ..  ..@ family       : NULL
 ..  ..@ face         : NULL
 ..  ..@ italic       : chr NA
 ..  ..@ fontweight   : num NA
 ..  ..@ fontwidth    : num NA
 ..  ..@ colour       : NULL
 ..  ..@ size         : NULL
 ..  ..@ hjust        : NULL
 ..  ..@ vjust        : NULL
 ..  ..@ angle        : NULL
 ..  ..@ lineheight   : NULL
 ..  ..@ margin       : <ggplot2::margin> num [1:4] 0 4.95 0 0
 ..  ..@ debug        : NULL
 ..  ..@ inherit.blank: logi TRUE
 .. $ axis.text.y.right               : <ggplot2::element_text>
 ..  ..@ family       : NULL
 ..  ..@ face         : NULL
 ..  ..@ italic       : chr NA
 ..  ..@ fontweight   : num NA
 ..  ..@ fontwidth    : num NA
 ..  ..@ colour       : NULL
 ..  ..@ size         : NULL
 ..  ..@ hjust        : NULL
 ..  ..@ vjust        : NULL
 ..  ..@ angle        : NULL
 ..  ..@ lineheight   : NULL
 ..  ..@ margin       : <ggplot2::margin> num [1:4] 0 0 0 4.95
 ..  ..@ debug        : NULL
 ..  ..@ inherit.blank: logi TRUE
 .. $ axis.text.theta                 : NULL
 .. $ axis.text.r                     : <ggplot2::element_text>
 ..  ..@ family       : NULL
 ..  ..@ face         : NULL
 ..  ..@ italic       : chr NA
 ..  ..@ fontweight   : num NA
 ..  ..@ fontwidth    : num NA
 ..  ..@ colour       : NULL
 ..  ..@ size         : NULL
 ..  ..@ hjust        : num 0.5
 ..  ..@ vjust        : NULL
 ..  ..@ angle        : NULL
 ..  ..@ lineheight   : NULL
 ..  ..@ margin       : <ggplot2::margin> num [1:4] 0 2.2 0 2.2
 ..  ..@ debug        : NULL
 ..  ..@ inherit.blank: logi TRUE
 .. $ axis.ticks                      : <ggplot2::element_blank>
 .. $ axis.ticks.x                    : NULL
 .. $ axis.ticks.x.top                : NULL
 .. $ axis.ticks.x.bottom             : NULL
 .. $ axis.ticks.y                    : NULL
 .. $ axis.ticks.y.left               : NULL
 .. $ axis.ticks.y.right              : NULL
 .. $ axis.ticks.theta                : NULL
 .. $ axis.ticks.r                    : NULL
 .. $ axis.minor.ticks.x.top          : NULL
 .. $ axis.minor.ticks.x.bottom       : NULL
 .. $ axis.minor.ticks.y.left         : NULL
 .. $ axis.minor.ticks.y.right        : NULL
 .. $ axis.minor.ticks.theta          : NULL
 .. $ axis.minor.ticks.r              : NULL
 .. $ axis.ticks.length               : 'rel' num 0.5
 .. $ axis.ticks.length.x             : NULL
 .. $ axis.ticks.length.x.top         : NULL
 .. $ axis.ticks.length.x.bottom      : NULL
 .. $ axis.ticks.length.y             : NULL
 .. $ axis.ticks.length.y.left        : NULL
 .. $ axis.ticks.length.y.right       : NULL
 .. $ axis.ticks.length.theta         : NULL
 .. $ axis.ticks.length.r             : NULL
 .. $ axis.minor.ticks.length         : 'rel' num 0.75
 .. $ axis.minor.ticks.length.x       : NULL
 .. $ axis.minor.ticks.length.x.top   : NULL
 .. $ axis.minor.ticks.length.x.bottom: NULL
 .. $ axis.minor.ticks.length.y       : NULL
 .. $ axis.minor.ticks.length.y.left  : NULL
 .. $ axis.minor.ticks.length.y.right : NULL
 .. $ axis.minor.ticks.length.theta   : NULL
 .. $ axis.minor.ticks.length.r       : NULL
 .. $ axis.line                       : <ggplot2::element_blank>
 .. $ axis.line.x                     : NULL
 .. $ axis.line.x.top                 : NULL
 .. $ axis.line.x.bottom              : NULL
 .. $ axis.line.y                     : NULL
 .. $ axis.line.y.left                : NULL
 .. $ axis.line.y.right               : NULL
 .. $ axis.line.theta                 : NULL
 .. $ axis.line.r                     : NULL
 .. $ legend.background               : <ggplot2::element_blank>
 .. $ legend.margin                   : NULL
 .. $ legend.spacing                  : 'rel' num 2
 .. $ legend.spacing.x                : NULL
 .. $ legend.spacing.y                : NULL
 .. $ legend.key                      : <ggplot2::element_blank>
 .. $ legend.key.size                 : 'simpleUnit' num 1.2lines
 ..  ..- attr(*, "unit")= int 3
 .. $ legend.key.height               : NULL
 .. $ legend.key.width                : NULL
 .. $ legend.key.spacing              : NULL
 .. $ legend.key.spacing.x            : NULL
 .. $ legend.key.spacing.y            : NULL
 .. $ legend.key.justification        : NULL
 .. $ legend.frame                    : NULL
 .. $ legend.ticks                    : NULL
 .. $ legend.ticks.length             : 'rel' num 0.2
 .. $ legend.axis.line                : NULL
 .. $ legend.text                     : <ggplot2::element_text>
 ..  ..@ family       : NULL
 ..  ..@ face         : NULL
 ..  ..@ italic       : chr NA
 ..  ..@ fontweight   : num NA
 ..  ..@ fontwidth    : num NA
 ..  ..@ colour       : NULL
 ..  ..@ size         : 'rel' num 0.8
 ..  ..@ hjust        : NULL
 ..  ..@ vjust        : NULL
 ..  ..@ angle        : NULL
 ..  ..@ lineheight   : NULL
 ..  ..@ margin       : NULL
 ..  ..@ debug        : NULL
 ..  ..@ inherit.blank: logi TRUE
 .. $ legend.text.position            : NULL
 .. $ legend.title                    : <ggplot2::element_text>
 ..  ..@ family       : NULL
 ..  ..@ face         : NULL
 ..  ..@ italic       : chr NA
 ..  ..@ fontweight   : num NA
 ..  ..@ fontwidth    : num NA
 ..  ..@ colour       : NULL
 ..  ..@ size         : NULL
 ..  ..@ hjust        : num 0
 ..  ..@ vjust        : NULL
 ..  ..@ angle        : NULL
 ..  ..@ lineheight   : NULL
 ..  ..@ margin       : NULL
 ..  ..@ debug        : NULL
 ..  ..@ inherit.blank: logi TRUE
 .. $ legend.title.position           : NULL
 .. $ legend.position                 : chr "right"
 .. $ legend.position.inside          : NULL
 .. $ legend.direction                : NULL
 .. $ legend.byrow                    : NULL
 .. $ legend.justification            : chr "center"
 .. $ legend.justification.top        : NULL
 .. $ legend.justification.bottom     : NULL
 .. $ legend.justification.left       : NULL
 .. $ legend.justification.right      : NULL
 .. $ legend.justification.inside     : NULL
 ..  [list output truncated]
  ..@ complete: logi TRUE
  ..@ validate: logi TRUE
 @ complete: logi FALSE
 @ validate: logi TRUE
\end{verbatim}

\begin{Shaded}
\begin{Highlighting}[]
\NormalTok{effects\_block }\OtherTok{\textless{}{-}} \FunctionTok{map\_dfr}\NormalTok{(}
  \FunctionTok{seq\_along}\NormalTok{(models\_block),}
  \SpecialCharTok{\textasciitilde{}}\NormalTok{ \{}
\NormalTok{    coef\_table }\OtherTok{\textless{}{-}} \FunctionTok{summary}\NormalTok{(models\_block[[.x]])}\SpecialCharTok{$}\NormalTok{coefficients}
    \FunctionTok{data.frame}\NormalTok{(}
      \AttributeTok{term =} \FunctionTok{rownames}\NormalTok{(coef\_table),}
      \AttributeTok{estimate =}\NormalTok{ coef\_table[, }\StringTok{"Estimate"}\NormalTok{],}
      \AttributeTok{std.error =}\NormalTok{ coef\_table[, }\StringTok{"Std. Error"}\NormalTok{],}
      \AttributeTok{statistic =}\NormalTok{ coef\_table[, }\StringTok{"t value"}\NormalTok{],}
      \AttributeTok{p.value =}\NormalTok{ coef\_table[, }\StringTok{"Pr(\textgreater{}|t|)"}\NormalTok{],}
      \AttributeTok{block =}\NormalTok{ .x }\SpecialCharTok{{-}} \DecValTok{1}\NormalTok{,  }\CommentTok{\# blocks are 0{-}indexed (0{-}19)}
      \AttributeTok{block\_c =}\NormalTok{ (.x }\SpecialCharTok{{-}} \DecValTok{1}\NormalTok{) }\SpecialCharTok{{-}} \FloatTok{9.5}  \CommentTok{\# centered around mean}
\NormalTok{    ) }\SpecialCharTok{\%\textgreater{}\%}
      \FunctionTok{filter}\NormalTok{(term }\SpecialCharTok{!=} \StringTok{"(Intercept)"}\NormalTok{)}
\NormalTok{  \}}
\NormalTok{) }\SpecialCharTok{\%\textgreater{}\%}
  \FunctionTok{mutate}\NormalTok{(}
    \AttributeTok{sig\_label =} \FunctionTok{case\_when}\NormalTok{(}
\NormalTok{      p.value }\SpecialCharTok{\textless{}} \FloatTok{0.001} \SpecialCharTok{\textasciitilde{}} \StringTok{"***"}\NormalTok{,}
\NormalTok{      p.value }\SpecialCharTok{\textless{}} \FloatTok{0.01} \SpecialCharTok{\textasciitilde{}} \StringTok{"**"}\NormalTok{,}
\NormalTok{      p.value }\SpecialCharTok{\textless{}} \FloatTok{0.05} \SpecialCharTok{\textasciitilde{}} \StringTok{"*"}\NormalTok{,}
      \ConstantTok{TRUE} \SpecialCharTok{\textasciitilde{}} \StringTok{""}
\NormalTok{    ))}

\NormalTok{p\_h1h2\_fixed\_effects }\OtherTok{\textless{}{-}} \FunctionTok{ggplot}\NormalTok{(effects\_block, }\FunctionTok{aes}\NormalTok{(}\AttributeTok{x =}\NormalTok{ block\_c, }\AttributeTok{y =}\NormalTok{ estimate, }\AttributeTok{color =}\NormalTok{ term, }\AttributeTok{group =}\NormalTok{ term)) }\SpecialCharTok{+}
    \FunctionTok{geom\_line}\NormalTok{() }\SpecialCharTok{+}
    \FunctionTok{geom\_point}\NormalTok{() }\SpecialCharTok{+}
    \FunctionTok{geom\_errorbar}\NormalTok{(}\FunctionTok{aes}\NormalTok{(}\AttributeTok{ymin =}\NormalTok{ estimate }\SpecialCharTok{{-}}\NormalTok{ std.error, }\AttributeTok{ymax =}\NormalTok{ estimate }\SpecialCharTok{+}\NormalTok{ std.error), }\AttributeTok{width =} \FloatTok{0.2}\NormalTok{) }\SpecialCharTok{+}
    \FunctionTok{geom\_text}\NormalTok{(}\FunctionTok{aes}\NormalTok{(}\AttributeTok{label =}\NormalTok{ sig\_label), }\AttributeTok{vjust =} \SpecialCharTok{{-}}\FloatTok{1.2}\NormalTok{, }\AttributeTok{size =} \DecValTok{5}\NormalTok{, }\AttributeTok{show.legend =} \ConstantTok{FALSE}\NormalTok{) }\SpecialCharTok{+}
    \FunctionTok{labs}\NormalTok{(}\AttributeTok{x =} \StringTok{"Block (zero{-}centered)"}\NormalTok{, }\AttributeTok{y =} \StringTok{"Effect Estimate"}\NormalTok{, }\AttributeTok{color =} \StringTok{"Parameter"}\NormalTok{) }\SpecialCharTok{+}
    \FunctionTok{theme\_minimal}\NormalTok{()}

\FunctionTok{ggsave}\NormalTok{(p\_h1h2\_fixed\_effects, }\AttributeTok{file =} \StringTok{"../plots/h1h2\_fixed\_effects.png"}\NormalTok{)}
\end{Highlighting}
\end{Shaded}

\begin{verbatim}
Saving 5.5 x 3.5 in image
\end{verbatim}

We first determined the optimal effects structure using final-block
data, comparing candidate models by summing AIC values across all five
matrix sizes. This comparison favored the maximal random effects
structure
\texttt{(surprisal\ ×\ prev\_entropy\ \textbar{}\textbar{}\ subject)\ +\ (1\ \textbar{}\ position)}
over simpler alternatives, and the full fixed effects model
\texttt{surprisal\ ×\ prev\_entropy\ +\ is\_repetition} over nested
versions. This structure converged without singularity across all 100
models (20 blocks × 5 conditions), and is therefore used for all
subsequent analyses.

To examine learning trajectories, we fit this selected model structure
separately to each block within each matrix size condition, yielding 100
models total (20 blocks × 5 conditions). This approach allows us to
track how sensitivity to surprisal and entropy emerges and changes over
the course of learning, and whether these trajectories differ across
state space sizes.

\subsection{State Space Size Effect}\label{state-space-size-effect}

\begin{Shaded}
\begin{Highlighting}[]
\CommentTok{\# Extract coefficients from all per{-}block models across all conditions}
\CommentTok{\# This creates a dataframe with: matrix\_size, block, and coefficient estimates}

\FunctionTok{library}\NormalTok{(broom.mixed)}

\CommentTok{\# Get all per{-}block model directories}
\NormalTok{condition\_dirs }\OtherTok{\textless{}{-}} \FunctionTok{list.files}\NormalTok{(}\StringTok{"models/"}\NormalTok{, }\AttributeTok{pattern =} \StringTok{"per\_block\_condition\_"}\NormalTok{, }\AttributeTok{full.names =} \ConstantTok{TRUE}\NormalTok{)}

\CommentTok{\# Extract coefficients from each condition\textquotesingle{}s per{-}block models}
\NormalTok{coef\_df }\OtherTok{\textless{}{-}} \FunctionTok{map\_dfr}\NormalTok{(condition\_dirs, }\ControlFlowTok{function}\NormalTok{(dir) \{}
\NormalTok{    condition\_num }\OtherTok{\textless{}{-}} \FunctionTok{gsub}\NormalTok{(}\StringTok{".*per\_block\_condition\_(}\SpecialCharTok{\textbackslash{}\textbackslash{}}\StringTok{d+)"}\NormalTok{, }\StringTok{"}\SpecialCharTok{\textbackslash{}\textbackslash{}}\StringTok{1"}\NormalTok{, dir)}
\NormalTok{    matrix\_size }\OtherTok{\textless{}{-}} \FunctionTok{paste0}\NormalTok{(condition\_num, }\StringTok{"x"}\NormalTok{, condition\_num)}

    \CommentTok{\# Get all model files in this directory}
\NormalTok{    model\_files }\OtherTok{\textless{}{-}} \FunctionTok{list.files}\NormalTok{(dir, }\AttributeTok{pattern =} \StringTok{"model\_block\_.*}\SpecialCharTok{\textbackslash{}\textbackslash{}}\StringTok{.rds"}\NormalTok{, }\AttributeTok{full.names =} \ConstantTok{TRUE}\NormalTok{)}

    \CommentTok{\# Extract block number and coefficients from each model}
    \FunctionTok{map\_dfr}\NormalTok{(model\_files, }\ControlFlowTok{function}\NormalTok{(file) \{}
\NormalTok{        block\_num }\OtherTok{\textless{}{-}} \FunctionTok{as.numeric}\NormalTok{(}\FunctionTok{gsub}\NormalTok{(}\StringTok{".*model\_block\_(}\SpecialCharTok{\textbackslash{}\textbackslash{}}\StringTok{d+)}\SpecialCharTok{\textbackslash{}\textbackslash{}}\StringTok{.rds"}\NormalTok{, }\StringTok{"}\SpecialCharTok{\textbackslash{}\textbackslash{}}\StringTok{1"}\NormalTok{, file))}

        \CommentTok{\# Load model and extract fixed effects}
\NormalTok{        m }\OtherTok{\textless{}{-}} \FunctionTok{readRDS}\NormalTok{(file)}
\NormalTok{        coefs }\OtherTok{\textless{}{-}} \FunctionTok{fixef}\NormalTok{(m)}

        \CommentTok{\# Return as a row}
        \FunctionTok{data.frame}\NormalTok{(}
            \AttributeTok{matrix\_size =}\NormalTok{ matrix\_size,}
            \AttributeTok{matrix\_size\_num =} \FunctionTok{as.numeric}\NormalTok{(condition\_num),}
            \AttributeTok{matrix\_size\_c =} \FunctionTok{as.numeric}\NormalTok{(condition\_num) }\SpecialCharTok{{-}} \DecValTok{6}\NormalTok{,}
            \AttributeTok{block =}\NormalTok{ block\_num,}
            \AttributeTok{block\_c =}\NormalTok{ block\_num }\SpecialCharTok{{-}} \FloatTok{9.5}\NormalTok{,}
            \AttributeTok{intercept =}\NormalTok{ coefs[}\StringTok{"(Intercept)"}\NormalTok{],}
            \AttributeTok{surprisal =}\NormalTok{ coefs[}\StringTok{"surprisal\_c"}\NormalTok{],}
            \AttributeTok{prev\_entropy =}\NormalTok{ coefs[}\StringTok{"prev\_entropy\_c"}\NormalTok{],}
            \AttributeTok{interaction =}\NormalTok{ coefs[}\StringTok{"surprisal\_c:prev\_entropy\_c"}\NormalTok{],}
            \AttributeTok{is\_repetition =}\NormalTok{ coefs[}\StringTok{"is\_repetitionTRUE"}\NormalTok{]}
\NormalTok{        )}
\NormalTok{    \})}
\NormalTok{\})}

\CommentTok{\# View structure}
\FunctionTok{names}\NormalTok{(coef\_df)}
\FunctionTok{head}\NormalTok{(coef\_df)}
\FunctionTok{summary}\NormalTok{(coef\_df)}
\end{Highlighting}
\end{Shaded}

\begin{Shaded}
\begin{Highlighting}[]
\CommentTok{\# H3: Test whether surprisal, entropy, and interaction effects}
\CommentTok{\# are moderated by matrix\_size and block}

\CommentTok{\# Model 1: Surprisal effect \textasciitilde{} matrix\_size * block}
\NormalTok{m\_h3\_surprisal }\OtherTok{\textless{}{-}} \FunctionTok{lm}\NormalTok{(surprisal }\SpecialCharTok{\textasciitilde{}}\NormalTok{ matrix\_size\_num }\SpecialCharTok{*}\NormalTok{ block\_c, }\AttributeTok{data =}\NormalTok{ coef\_df)}
\FunctionTok{summary}\NormalTok{(m\_h3\_surprisal)}

\CommentTok{\# Model 2: Previous entropy effect \textasciitilde{} matrix\_size * block}
\NormalTok{m\_h3\_entropy }\OtherTok{\textless{}{-}} \FunctionTok{lm}\NormalTok{(prev\_entropy }\SpecialCharTok{\textasciitilde{}}\NormalTok{ matrix\_size\_num }\SpecialCharTok{*}\NormalTok{ block\_c, }\AttributeTok{data =}\NormalTok{ coef\_df)}
\FunctionTok{summary}\NormalTok{(m\_h3\_entropy)}

\CommentTok{\# Model 3: Interaction effect \textasciitilde{} matrix\_size * block}
\NormalTok{m\_h3\_interaction }\OtherTok{\textless{}{-}} \FunctionTok{lm}\NormalTok{(interaction }\SpecialCharTok{\textasciitilde{}}\NormalTok{ matrix\_size\_num }\SpecialCharTok{*}\NormalTok{ block\_c, }\AttributeTok{data =}\NormalTok{ coef\_df)}
\FunctionTok{summary}\NormalTok{(m\_h3\_interaction)}

\CommentTok{\# Model 4: Repetition effect \textasciitilde{} matrix\_size * block}
\NormalTok{m\_h3\_repetition }\OtherTok{\textless{}{-}} \FunctionTok{lm}\NormalTok{(is\_repetition }\SpecialCharTok{\textasciitilde{}}\NormalTok{ matrix\_size\_num }\SpecialCharTok{*}\NormalTok{ block\_c, }\AttributeTok{data =}\NormalTok{ coef\_df)}
\FunctionTok{summary}\NormalTok{(m\_h3\_repetition)}
\end{Highlighting}
\end{Shaded}

\begin{Shaded}
\begin{Highlighting}[]
\CommentTok{\# Create plots directory if it doesn\textquotesingle{}t exist}
\FunctionTok{dir.create}\NormalTok{(}\StringTok{"plots"}\NormalTok{, }\AttributeTok{showWarnings =} \ConstantTok{FALSE}\NormalTok{)}

\CommentTok{\# Visualize how effects change across matrix size and block}

\CommentTok{\# Plot surprisal effect}
\NormalTok{p1 }\OtherTok{\textless{}{-}} \FunctionTok{ggplot}\NormalTok{(coef\_df, }\FunctionTok{aes}\NormalTok{(}\AttributeTok{x =}\NormalTok{ block\_c, }\AttributeTok{y =}\NormalTok{ surprisal, }\AttributeTok{color =}\NormalTok{ matrix\_size\_c, }\AttributeTok{group =}\NormalTok{ matrix\_size\_c)) }\SpecialCharTok{+}
    \FunctionTok{geom\_smooth}\NormalTok{(}\AttributeTok{method =} \StringTok{"lm"}\NormalTok{, }\AttributeTok{se =} \ConstantTok{TRUE}\NormalTok{, }\AttributeTok{alpha =} \FloatTok{0.2}\NormalTok{) }\SpecialCharTok{+}
    \FunctionTok{geom\_point}\NormalTok{(}\AttributeTok{alpha =} \FloatTok{0.3}\NormalTok{) }\SpecialCharTok{+}
    \FunctionTok{labs}\NormalTok{(}
        \AttributeTok{title =} \StringTok{"H1: Surprisal Effect Across Learning"}\NormalTok{,}
        \AttributeTok{x =} \StringTok{"Block (zero{-}centered)"}\NormalTok{, }\AttributeTok{y =} \StringTok{"Surprisal Coefficient"}\NormalTok{, }\AttributeTok{color =} \StringTok{"Matrix Size (zero{-}centered)"}
\NormalTok{    ) }\SpecialCharTok{+}
    \FunctionTok{theme\_minimal}\NormalTok{()}
\FunctionTok{ggsave}\NormalTok{(}\StringTok{"../plots/h3\_surprisal\_by\_block.png"}\NormalTok{, p1, }\AttributeTok{width =} \DecValTok{8}\NormalTok{, }\AttributeTok{height =} \DecValTok{6}\NormalTok{, }\AttributeTok{dpi =} \DecValTok{300}\NormalTok{)}
\FunctionTok{print}\NormalTok{(p1)}
\end{Highlighting}
\end{Shaded}

\begin{Shaded}
\begin{Highlighting}[]
\CommentTok{\# Plot entropy effect}
\NormalTok{p2 }\OtherTok{\textless{}{-}} \FunctionTok{ggplot}\NormalTok{(coef\_df, }\FunctionTok{aes}\NormalTok{(}\AttributeTok{x =}\NormalTok{ block\_c, }\AttributeTok{y =}\NormalTok{ prev\_entropy, }\AttributeTok{color =}\NormalTok{ matrix\_size\_c, }\AttributeTok{group =}\NormalTok{ matrix\_size\_c)) }\SpecialCharTok{+}
    \FunctionTok{geom\_smooth}\NormalTok{(}\AttributeTok{method =} \StringTok{"lm"}\NormalTok{, }\AttributeTok{se =} \ConstantTok{TRUE}\NormalTok{, }\AttributeTok{alpha =} \FloatTok{0.2}\NormalTok{) }\SpecialCharTok{+}
    \FunctionTok{geom\_point}\NormalTok{(}\AttributeTok{alpha =} \FloatTok{0.3}\NormalTok{) }\SpecialCharTok{+}
    \FunctionTok{labs}\NormalTok{(}
        \AttributeTok{title =} \StringTok{"H2: Entropy Effect Across Learning"}\NormalTok{,}
        \AttributeTok{x =} \StringTok{"Block (zero{-}centered)"}\NormalTok{, }\AttributeTok{y =} \StringTok{"Previous Entropy Coefficient"}\NormalTok{, }\AttributeTok{color =} \StringTok{"Matrix Size (zero{-}centered)"}
\NormalTok{    ) }\SpecialCharTok{+}
    \FunctionTok{theme\_minimal}\NormalTok{()}
\FunctionTok{ggsave}\NormalTok{(}\StringTok{"../plots/h3\_entropy\_by\_block.png"}\NormalTok{, p2, }\AttributeTok{width =} \DecValTok{8}\NormalTok{, }\AttributeTok{height =} \DecValTok{6}\NormalTok{, }\AttributeTok{dpi =} \DecValTok{300}\NormalTok{)}
\FunctionTok{print}\NormalTok{(p2)}
\end{Highlighting}
\end{Shaded}

\begin{Shaded}
\begin{Highlighting}[]
\CommentTok{\# Plot interaction effect}
\NormalTok{p3 }\OtherTok{\textless{}{-}} \FunctionTok{ggplot}\NormalTok{(coef\_df, }\FunctionTok{aes}\NormalTok{(}\AttributeTok{x =}\NormalTok{ block\_c, }\AttributeTok{y =}\NormalTok{ interaction, }\AttributeTok{color =}\NormalTok{ matrix\_size\_c, }\AttributeTok{group =}\NormalTok{ matrix\_size\_c)) }\SpecialCharTok{+}
    \FunctionTok{geom\_smooth}\NormalTok{(}\AttributeTok{method =} \StringTok{"lm"}\NormalTok{, }\AttributeTok{se =} \ConstantTok{TRUE}\NormalTok{, }\AttributeTok{alpha =} \FloatTok{0.2}\NormalTok{) }\SpecialCharTok{+}
    \FunctionTok{geom\_point}\NormalTok{(}\AttributeTok{alpha =} \FloatTok{0.3}\NormalTok{) }\SpecialCharTok{+}
    \FunctionTok{labs}\NormalTok{(}
        \AttributeTok{title =} \StringTok{"H2: Surprisal × Entropy Interaction Across Learning"}\NormalTok{,}
        \AttributeTok{x =} \StringTok{"Block (zero{-}centered)"}\NormalTok{, }\AttributeTok{y =} \StringTok{"Interaction Coefficient"}\NormalTok{, }\AttributeTok{color =} \StringTok{"Matrix Size (zero{-}centered)"}
\NormalTok{    ) }\SpecialCharTok{+}
    \FunctionTok{theme\_minimal}\NormalTok{()}
\FunctionTok{ggsave}\NormalTok{(}\StringTok{"../plots/h3\_interaction\_by\_block.png"}\NormalTok{, p3, }\AttributeTok{width =} \DecValTok{8}\NormalTok{, }\AttributeTok{height =} \DecValTok{6}\NormalTok{, }\AttributeTok{dpi =} \DecValTok{300}\NormalTok{)}
\FunctionTok{print}\NormalTok{(p3)}
\end{Highlighting}
\end{Shaded}

\begin{Shaded}
\begin{Highlighting}[]
\CommentTok{\# Plot matrix size effect (averaging across blocks)}
\NormalTok{coef\_summary }\OtherTok{\textless{}{-}}\NormalTok{ coef\_df }\SpecialCharTok{\%\textgreater{}\%}
    \FunctionTok{group\_by}\NormalTok{(matrix\_size\_c, matrix\_size\_num) }\SpecialCharTok{\%\textgreater{}\%}
    \FunctionTok{summarize}\NormalTok{(}
        \AttributeTok{surprisal\_mean =} \FunctionTok{mean}\NormalTok{(surprisal, }\AttributeTok{na.rm =} \ConstantTok{TRUE}\NormalTok{),}
        \AttributeTok{surprisal\_se =} \FunctionTok{sd}\NormalTok{(surprisal, }\AttributeTok{na.rm =} \ConstantTok{TRUE}\NormalTok{) }\SpecialCharTok{/} \FunctionTok{sqrt}\NormalTok{(}\FunctionTok{n}\NormalTok{()),}
        \AttributeTok{entropy\_mean =} \FunctionTok{mean}\NormalTok{(prev\_entropy, }\AttributeTok{na.rm =} \ConstantTok{TRUE}\NormalTok{),}
        \AttributeTok{entropy\_se =} \FunctionTok{sd}\NormalTok{(prev\_entropy, }\AttributeTok{na.rm =} \ConstantTok{TRUE}\NormalTok{) }\SpecialCharTok{/} \FunctionTok{sqrt}\NormalTok{(}\FunctionTok{n}\NormalTok{()),}
        \AttributeTok{interaction\_mean =} \FunctionTok{mean}\NormalTok{(interaction, }\AttributeTok{na.rm =} \ConstantTok{TRUE}\NormalTok{),}
        \AttributeTok{interaction\_se =} \FunctionTok{sd}\NormalTok{(interaction, }\AttributeTok{na.rm =} \ConstantTok{TRUE}\NormalTok{) }\SpecialCharTok{/} \FunctionTok{sqrt}\NormalTok{(}\FunctionTok{n}\NormalTok{())}
\NormalTok{    )}

\NormalTok{p4 }\OtherTok{\textless{}{-}} \FunctionTok{ggplot}\NormalTok{(coef\_summary, }\FunctionTok{aes}\NormalTok{(}\AttributeTok{x =}\NormalTok{ matrix\_size\_num)) }\SpecialCharTok{+}
    \FunctionTok{geom\_line}\NormalTok{(}\FunctionTok{aes}\NormalTok{(}\AttributeTok{y =}\NormalTok{ surprisal\_mean, }\AttributeTok{color =} \StringTok{"Surprisal"}\NormalTok{)) }\SpecialCharTok{+}
    \FunctionTok{geom\_point}\NormalTok{(}\FunctionTok{aes}\NormalTok{(}\AttributeTok{y =}\NormalTok{ surprisal\_mean, }\AttributeTok{color =} \StringTok{"Surprisal"}\NormalTok{)) }\SpecialCharTok{+}
    \FunctionTok{geom\_errorbar}\NormalTok{(}
        \FunctionTok{aes}\NormalTok{(}
            \AttributeTok{ymin =}\NormalTok{ surprisal\_mean }\SpecialCharTok{{-}}\NormalTok{ surprisal\_se,}
            \AttributeTok{ymax =}\NormalTok{ surprisal\_mean }\SpecialCharTok{+}\NormalTok{ surprisal\_se, }\AttributeTok{color =} \StringTok{"Surprisal"}
\NormalTok{        ),}
        \AttributeTok{width =} \FloatTok{0.2}
\NormalTok{    ) }\SpecialCharTok{+}
    \FunctionTok{geom\_line}\NormalTok{(}\FunctionTok{aes}\NormalTok{(}\AttributeTok{y =}\NormalTok{ entropy\_mean, }\AttributeTok{color =} \StringTok{"Entropy"}\NormalTok{)) }\SpecialCharTok{+}
    \FunctionTok{geom\_point}\NormalTok{(}\FunctionTok{aes}\NormalTok{(}\AttributeTok{y =}\NormalTok{ entropy\_mean, }\AttributeTok{color =} \StringTok{"Entropy"}\NormalTok{)) }\SpecialCharTok{+}
    \FunctionTok{geom\_errorbar}\NormalTok{(}
        \FunctionTok{aes}\NormalTok{(}
            \AttributeTok{ymin =}\NormalTok{ entropy\_mean }\SpecialCharTok{{-}}\NormalTok{ entropy\_se,}
            \AttributeTok{ymax =}\NormalTok{ entropy\_mean }\SpecialCharTok{+}\NormalTok{ entropy\_se, }\AttributeTok{color =} \StringTok{"Entropy"}
\NormalTok{        ),}
        \AttributeTok{width =} \FloatTok{0.2}
\NormalTok{    ) }\SpecialCharTok{+}
    \FunctionTok{labs}\NormalTok{(}
        \AttributeTok{title =} \StringTok{"H3: Effect Size by Matrix Size (Averaged Across Blocks)"}\NormalTok{,}
        \AttributeTok{x =} \StringTok{"Matrix Size (zero{-}centered)"}\NormalTok{, }\AttributeTok{y =} \StringTok{"Coefficient Estimate"}\NormalTok{, }\AttributeTok{color =} \StringTok{"Effect"}
\NormalTok{    ) }\SpecialCharTok{+}
    \FunctionTok{theme\_minimal}\NormalTok{()}
\FunctionTok{ggsave}\NormalTok{(}\StringTok{"../plots/h3\_effects\_by\_matrix\_size.png"}\NormalTok{, p4, }\AttributeTok{width =} \DecValTok{8}\NormalTok{, }\AttributeTok{height =} \DecValTok{6}\NormalTok{, }\AttributeTok{dpi =} \DecValTok{300}\NormalTok{)}
\FunctionTok{print}\NormalTok{(p4)}
\end{Highlighting}
\end{Shaded}

\section{Discussion}\label{discussion}

\section{References}\label{references}

\phantomsection\label{refs}
\begin{CSLReferences}{1}{0}
\bibitem[\citeproctext]{ref-farkas_complexity_2024}
Farkas, B. C., Krajcsi, A., Janacsek, K., \& Nemeth, D. (2024). The
complexity of measuring reliability in learning tasks: {An} illustration
using the {Alternating} {Serial} {Reaction} {Time} {Task}.
\emph{Behavior Research Methods}, \emph{56}(1), 301--317.
\url{https://doi.org/10.3758/s13428-022-02038-5}

\bibitem[\citeproctext]{ref-hedenius_grammar_2011}
Hedenius, M., Persson, J., Tremblay, A., Adi-Japha, E., Veríssimo, J.,
Dye, C. D., Alm, P., Jennische, M., Bruce Tomblin, J., \& Ullman, M. T.
(2011). Grammar predicts procedural learning and consolidation deficits
in children with {Specific} {Language} {Impairment}. \emph{Research in
Developmental Disabilities}, \emph{32}(6), 2362--2375.
\url{https://doi.org/10.1016/j.ridd.2011.07.026}

\bibitem[\citeproctext]{ref-howard_implicit_2004}
Howard, D. V., Howard, J. H., Japikse, K., DiYanni, C., Thompson, A., \&
Somberg, R. (2004). Implicit {Sequence} {Learning}: {Effects} of {Level}
of {Structure}, {Adult} {Age}, and {Extended} {Practice}.
\emph{Psychology and Aging}, \emph{19}(1), 79--92.
\url{https://doi.org/10.1037/0882-7974.19.1.79}

\bibitem[\citeproctext]{ref-howard_jr_age_1997}
Howard Jr., J. H., \& Howard, D. V. (1997). Age differences in implicit
learning of higher order dependencies in serial patterns.
\emph{Psychology and Aging}, \emph{12}(4), 634--656.
\url{https://doi.org/10.1037/0882-7974.12.4.634}

\bibitem[\citeproctext]{ref-janacsek_best_2012}
Janacsek, K., Fiser, J., \& Nemeth, D. (2012). The best time to acquire
new skills: Age-related differences in implicit sequence learning across
the human lifespan. \emph{Developmental Science}, \emph{15}(4),
496--505. \url{https://doi.org/10.1111/j.1467-7687.2012.01150.x}

\bibitem[\citeproctext]{ref-kassubek_changes_2001}
Kassubek, J., Schmidtke, K., Kimmig, H., Lücking, C. H., \& Greenlee, M.
W. (2001). Changes in cortical activation during mirror reading before
and after training: An {fMRI} study of procedural learning.
\emph{Cognitive Brain Research}, \emph{10}(3), 207--217.
https://doi.org/\url{https://doi.org/10.1016/S0926-6410(00)00037-9}

\bibitem[\citeproctext]{ref-kidd_individual_2016}
Kidd, E., \& Arciuli, J. (2016). Individual {Differences} in
{Statistical} {Learning} {Predict} {Children}'s {Comprehension} of
{Syntax}. \emph{Child Development}, \emph{87}(1), 184--193.
\url{https://doi.org/10.1111/cdev.12461}

\bibitem[\citeproctext]{ref-lammertink_statistical_2017}
Lammertink, I., Boersma, P., Wijnen, F., \& Rispens, J. (2017).
Statistical {Learning} in {Specific} {Language} {Impairment}: {A}
{Meta}-{Analysis}. \emph{Journal of Speech, Language, and Hearing
Research}, \emph{60}(12), 3474--3486.
\url{https://doi.org/10.1044/2017_JSLHR-L-16-0439}

\bibitem[\citeproctext]{ref-lum_procedural_2014}
Lum, J. A. G., Conti-Ramsden, G., Morgan, A. T., \& Ullman, M. T.
(2014). Procedural learning deficits in specific language impairment
({SLI}): {A} meta-analysis of serial reaction time task performance.
\emph{Cortex}, \emph{51}, 1--10.
\url{https://doi.org/10.1016/j.cortex.2013.10.011}

\bibitem[\citeproctext]{ref-lustig_transition_2022}
Lustig, C. (2022). \emph{The transition from implicit to explicit
knowledge representations in implicit sequence learning}
{[}Text.thesis.doctoral, Universität zu Köln{]}.
\url{http://www.uni-koeln.de/}

\bibitem[\citeproctext]{ref-mullaney_chinese_2017}
Mullaney, T. S. (2017). \emph{The {Chinese} {Typewriter}: {A}
{History}}. MIT Press.

\bibitem[\citeproctext]{ref-nissen_attentional_1987}
Nissen, M. J., \& Bullemer, P. (1987). Attentional requirements of
learning: {Evidence} from performance measures. \emph{Cognitive
Psychology}, \emph{19}(1), 1--32.
\url{https://doi.org/10.1016/0010-0285(87)90002-8}

\bibitem[\citeproctext]{ref-oliveira_reliability_2023}
Oliveira, C. M., Hayiou-Thomas, M. E., \& Henderson, L. M. (2023). The
reliability of the serial reaction time task: Meta-analysis of
test--retest correlations. \emph{Royal Society Open Science},
\emph{10}(7), 221542. \url{https://doi.org/10.1098/rsos.221542}

\bibitem[\citeproctext]{ref-song_perceptual_2008}
Song, S., Howard, J. H., \& Howard, D. V. (2008). Perceptual sequence
learning in a serial reaction time task. \emph{Experimental Brain
Research}, \emph{189}(2), 145--158.
\url{https://doi.org/10.1007/s00221-008-1411-z}

\bibitem[\citeproctext]{ref-squire_memory_2004}
Squire, L. R. (2004). Memory systems of the brain: {A} brief history and
current perspective. \emph{Neurobiology of Learning and Memory},
\emph{82}(3), 171--177.
https://doi.org/\url{https://doi.org/10.1016/j.nlm.2004.06.005}

\bibitem[\citeproctext]{ref-ullman_specific_2005}
Ullman, M. T., \& Pierpont, E. I. (2005). Specific {Language}
{Impairment} is not {Specific} to {Language}: The {Procedural} {Deficit}
{Hypothesis}. \emph{Cortex}, \emph{41}(3), 399--433.
\url{https://doi.org/10.1016/S0010-9452(08)70276-4}

\bibitem[\citeproctext]{ref-west_procedural_2018}
West, G., Vadillo, M. A., Shanks, D. R., \& Hulme, C. (2018). The
procedural learning deficit hypothesis of language learning disorders:
We see some problems. \emph{Developmental Science}, \emph{21}(2),
e12552. \url{https://doi.org/10.1111/desc.12552}

\end{CSLReferences}




\end{document}
